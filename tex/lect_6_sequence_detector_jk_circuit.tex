\documentclass[border=10pt]{standalone}
\usepackage{circuitikz}
\usetikzlibrary{calc}

\begin{document}
\begin{circuitikz}[american, scale=1.0]
    \ctikzset{logic ports=ieee}
    
    % Inputs
    \node (clk_in) at (-1, -5.1) {CLK};
    
    % Flip-Flops
    \node[flipflop JK] (JK_A) at (8, 2) {};
    \node[above] at (JK_A.n) {Flip-Flop A};
    \node[flipflop JK] (JK_B) at (8, -3) {};
    \node[above] at (JK_B.n) {Flip-Flop B};

    % Logic for JA = Bx'
    % AND gate (B, x')
    \path (JK_A.pin 1) -- ++(-3, 0) coordinate (ja_pos);
    \node[and port] (AND_JA) at (ja_pos) {$Bx'$};
    \draw (AND_JA.out) -- (JK_A.pin 1);

    % Inverter for x (to get x')
    % Place to left of AND_JA
    \draw (AND_JA.in 2) -- ++(-1, 0) node[not port, scale=0.5, anchor=out] (NOT_x) {};
    % Route x to NOT
    \draw (NOT_x.in) -- ++(-0.5,0) coordinate (x_bus_top);
    \draw (x_bus_top) to[short, *-] ++(-1,0) node[left] {$x$};

    % Logic for KA = x
    % Connect directly to x line
    \draw (JK_A.pin 3) to[short,-*] (JK_A.pin 3 -| x_bus_top);
    % We need to route x to here.
    % x_bus_top is at roughly (-4, 2-ish). KA is at (8, 2-offset).
    % Let's drop a line from x_bus_top down to KA level
    %\draw (x_bus_top) |- (KA_in) -- (JK_A.pin 3);

    % Logic for JB = x
    \draw (JK_B.pin 1) to[short,-*] (JK_B.pin 1 -| x_bus_top);
    %\draw (x_bus_top) |- (JB_in) -- (JK_B.pin 1);

    % Logic for KB = x'
    % Take from NOT_x output
    
    \path (NOT_x.out) -- ++(0.5,0) node[circ] (KB_out) {};
    \draw (JK_B.pin 3) -| (KB_out);
    
    % Connect x' to AND_JA (input 2) - already done via NOT_x placement

    % Feedback B to AND_JA (input 1)
    % B is at JK_B.pin 6
    \draw (JK_B.pin 6) -- ++(1, 0) coordinate (B_out);
    \node[right] at (B_out) {$B$};
    \path (JK_B.pin 6) -- ++(0.4, 0) coordinate (B_common);
    
    % Route B up to AND_JA
    \draw (B_common) to[short, *-] ++(0, 6.3) coordinate (B_top) -| (AND_JA.in 1);

    % Output Logic y = Ax
    \path (JK_A.pin 6) -- ++(2, 0) coordinate (y_pos);
    \node[and port, anchor=in 1] (AND_y) at (y_pos) {$Ax$};
    
    % A to AND_y
    \draw (JK_A.pin 6) -- ++(1, 0) coordinate (A_out);
    \node[above right] at (A_out) {$A$};
    \draw (A_out) -| (AND_y.in 1);
    
    % x to AND_y
    % Route x from far left
    \draw (x_bus_top) -- ++(0, -7.2) -| (AND_y.in 2);
    
    % Output y
    \draw (AND_y.out) -- ++(0.5, 0) node[right] {$y$};

    % Clock Distribution
    \draw (JK_B.pin 2) -- ++(-0.5, 0) node[circ] (clk_node) {};
    \draw (clk_in) -| (clk_node);
    \draw (clk_node) |- (JK_A.pin 2);

\end{circuitikz}
\end{document}
