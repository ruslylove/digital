\documentclass[border=10pt]{standalone}
\usepackage{circuitikz}

\ctikzset{
logic ports=ieee,
logic ports/scale=0.7,
}

\begin{document}
\begin{circuitikz}[]

    % ============================
    % 1. Place the XOR Gates (Cascaded)
    % ============================
    % Gate 1: Computes (x XOR y)
    \draw (2, 2) node[xor port] (XOR1) {};
    
    % Gate 2: Computes ((x XOR y) XOR z)
    \draw (5, 1) node[xor port] (XOR2) {};
    
    % Gate 3: Computes Final Result with P
    \draw (8, 0) node[xor port] (XOR3) {};


    % ============================
    % 2. Draw Inputs x and y
    % ============================
    \draw (XOR1.in 1) -- ++(-1.5, 0) node[left] {$x$};
    \draw (XOR1.in 2) -- ++(-1.5, 0) node[left] {$y$};


    % ============================
    % 3. First Inter-stage Connection & Input z
    % ============================
    % Connect Output 1 to Input 2 (Top)
    \draw (XOR1.out) -- ++(0.5,0) -| (XOR2.in 1);

    % Route Input z
    % Start it aligned on the left, route to XOR2 bottom input
    \draw (XOR2.in 2) -- ++(-4.5, 0) node[left] {$z$} ;


    % ============================
    % 4. Second Inter-stage Connection & Input P
    % ============================
    % Connect Output 2 to Input 3 (Top)
    \draw (XOR2.out) -- ++(0.5,0) -| (XOR3.in 1);

    % Route Input P (Parity Bit)
    % Start it aligned on the left, route to XOR3 bottom input
    \draw (XOR3.in 2) -- ++(-7.5, 0) node[left] {$P$}  ;


    % ============================
    % 5. Final Output C (Check)
    % ============================
    \draw (XOR3.out) -- ++(1, 0) node[right] {$C = x \oplus y \oplus z \oplus P$};

\end{circuitikz}
\end{document}