\documentclass[border=10pt]{standalone}
\usepackage{circuitikz}
\usetikzlibrary{calc}

\begin{document}
\begin{circuitikz}[american, scale=1.0]
    \ctikzset{logic ports=ieee}
    \ctikzset{flipflops/scale=0.85}
    
    % Signals
    \node (clk_in) at (-2, 0) {Clock};
    \node (logic_1) at (0, 2) {Logic 1};
    \node (clr_in) at (0, -3) {Reset};
    
    % Flip-Flops
    % LSB (A0) on Left -> MSB (A3) on Right
    % Ripple Counter using T Flip-Flops:
    % T connected to Logic 1 for toggle behavior.
    % Output Q connects to Clock of next stage (Negative Edge).
    
    \def\ffdist{4.0}
    
    \foreach \i [count=\n from 0] in {0,1,2,3} {
        \pgfmathsetmacro{\xpos}{\n*\ffdist}
        
        % T Flip-Flop
        \node[flipflop T] (FF\i) at (\xpos, 0) {};
        
        % Outputs
        \node[above] at (FF\i.n) {$FF_\i$};
        \draw (FF\i.pin 6) -- ++(0.5, 0) coordinate (Q\i_out);
        \draw (Q\i_out) -- ++(0.5, 0) node[circ, label={above:$A_\i$}] {};
        
        % T Input (pin 1) -> Logic 1
        \draw (FF\i.pin 1) -- ++(-0.5, 0) coordinate (T_stub);
        \draw (T_stub) -- (T_stub |- logic_1) node[circ] {};
        
        % Reset (Async Clear) - using down anchor
        \draw (FF\i.down) -- ++(0, -0.2) coordinate (clr_stub);
        \draw (clr_stub) -- (clr_stub |- clr_in) node[circ] {};
    }
    
    % Logic 1 Rail
    \draw (logic_1) -- (logic_1 -| FF3.pin 1);
    
    % Reset Rail
    \draw (clr_in) -- (clr_in -| FF3.down);
    
    % Clock Connections (Ripple)
    % A0 Clock -> External Clock
    \draw (clk_in) -- (FF0.pin 2);
    
    % Ripple: Q of previous -> Clock of next
    \draw (Q0_out) -- ++(0, -1.0) coordinate (rip0);
    \draw (rip0) -- (rip0 -| FF1.pin 2) -- (FF1.pin 2);
    
    \draw (Q1_out) -- ++(0, -1.0) coordinate (rip1);
    \draw (rip1) -- (rip1 -| FF2.pin 2) -- (FF2.pin 2);
    
    \draw (Q2_out) -- ++(0, -1.0) coordinate (rip2);
    \draw (rip2) -- (rip2 -| FF3.pin 2) -- (FF3.pin 2);
    
    % Bubbles on Clock Inputs (Negative Edge)
    \foreach \i in {0,1,2,3} {
        \draw ($(FF\i.pin 2)$) circle (0.1);
    }
    
\end{circuitikz}
\end{document}
