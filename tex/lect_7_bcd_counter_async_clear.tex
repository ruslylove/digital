\documentclass[border=10pt]{standalone}
\usepackage{circuitikz}
\usepackage{tikz}
\usetikzlibrary{calc, positioning}

\begin{document}
\begin{circuitikz}[american, scale=1.0]
    \ctikzset{logic ports=ieee}
    \tikzset{block/.style={draw, rectangle, minimum height=3.5cm, minimum width=3cm, align=center}}

    % --- Approach 2: Clear (Reset) ---
    % BCD Counter: Detect 10 (1010). Q3=1, Q1=1.
    
    \node [block] (cnt2) at (0, 0) {4-Bit\\Counter\\(Clear)};
    
    % Inputs/Outputs
    \node (clk2) at ($(cnt2.west) + (-1, -1.2)$) {CLK};
    \draw [->] (clk2) -- ($(cnt2.west) + (0, -1.2)$);
    
    \node (clr2) at ($(cnt2.west) + (-1, 0.5)$) {};
    \node [above=0.3 of cnt2.west][xshift=0.5cm] {Clear};
    % Logic for Clear
    \node [nand port, scale=0.8, anchor=out] (nand2) at ($(cnt2.west) + (-1.5, 0.5)$) {};
    \draw (nand2.out) -- ($(cnt2.west) + (0, 0.5)$);
    
    % 4-bit Outputs (Right)
    \foreach \i in {0,1,2,3} {
        \coordinate (Q2_\i_pin) at ($(cnt2.east) + (0, {-1.0 + \i*0.6})$);
        \draw (Q2_\i_pin) -- ++(1.5, 0) node[right] {$Q_\i$};
    }
    
    % Feedback from Q3 and Q1 (1010)
    \coordinate (target2Q3) at ($(cnt2.east) + (0, {-1.0 + 1*0.6})$); % Q3
    \coordinate (target2Q1) at ($(cnt2.east) + (0, {-1.0 + 3*0.6})$); % Q1
    
    \draw (target2Q3) ++(0.5, 0) coordinate (tap2_3);
    \node [circ] at (tap2_3) {};
    \draw (tap2_3) -- ++(0, 2.5) -| (nand2.in 1);
    
    \draw (target2Q1) ++(0.8, 0) coordinate (tap2_1); % Offset x
    \node [circ] at (tap2_1) {};
    \draw (tap2_1) -- ++(0, 2.0) -- ++(-7.5,0) |- (nand2.in 2);
    
    \node [below=0.2cm of nand2] {Detect 10 (1010)};
    
    %\node [below=2.0cm of cnt2] {(b) Using Asynchronous Clear (Detect 10)};

\end{circuitikz}
\end{document}
