\documentclass[border=10pt]{standalone}
\usepackage{tikz}
\usepackage{circuitikz}
\ctikzset{logic ports=ieee}

\begin{document}
\begin{circuitikz}
    
    % --- Frequency Divider (Counter) ---
    \draw (0,0) node[draw, rectangle, minimum width=2cm, minimum height=1.5cm, fill=yellow!10] (counter) {};
    \node[above, font=\footnotesize] at (counter.north) {Freq. Divider};
    
    % System CLK Input
    \draw (counter.west) to[short,-o] ++(-1,0) node[left] {System CLK};
    
    % Mod. CLK Output
    \draw (counter.east) -- ++(1,0) coordinate (mod_clk);
    \node[above, font=\tiny] at (mod_clk) {Mod. CLK};


    % --- State Minterm (Detection Logic) ---
    % Assume state q_n lines come in. Let's abstract this as a block or inputs.
    % The user said "final state minterm", implying logic that detects S6.
    
    \draw (4, 1.5) node[and port, number inputs=3] (minterm) {};
    \node[left, font=\footnotesize] at (minterm.in 1) {$Q_2$};
    \node[left, font=\footnotesize] at (minterm.in 2) {$Q_1$};
    \node[left, font=\footnotesize] at (minterm.in 3) {$\overline{Q_0}$};
    \node[above, font=\footnotesize] at (minterm.north) {Match ($S_6$)};
    
    % --- Final Output Modulation ---
    \draw ($(counter.east) + (5,0)$) node[and port,anchor=in 2] (final_and) {};
    
    % Connect Minterm to Final AND
    \draw (minterm.out) -| (final_and.in 1);
    
    % Connect Mod. CLK to Final AND
    \draw (mod_clk) |- (final_and.in 2);
    
    % Final Output
    \draw (final_and.out) to[short,-o] ++(0.5,0) node[right] {Detected};

\end{circuitikz}
\end{document}
