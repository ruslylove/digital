\documentclass[border=10pt]{standalone}
\usepackage{circuitikz}
\usetikzlibrary{shapes, arrows, positioning, calc}

\begin{document}
\begin{circuitikz}

    % Box for the Combinational Circuit
    \node[draw, rectangle, minimum width=4cm, minimum height=5cm, thick, fill=blue!10] (CC) at (0,0) {\large \begin{tabular}{c} Combinational \\ Logic \\ Circuit \end{tabular}};

    % Inputs
    \foreach \y/\i in {2/1, 1/2, -2/n} {
        \draw[thick] ($(CC.west)+(0, \y)$) -- ++(-1.5, 0) node[left] {$Input_{\i}$};
    }
    % Dots for inputs
    \node at ($(CC.west) + (-0.75, -0.2)$) {$\vdots$};


    % Outputs
    \foreach \y/\i in {2/1, 1/2, -2/m} {
        \draw[thick] ($(CC.east)+(0, \y)$) -- ++(1.5, 0) node[right] {$Output_{\i}$};
    }
    % Dots for outputs
    \node at ($(CC.east) + (0.75, -0.2)$) {$\vdots$};
    
    % Internal connection (optional visual flair)
    % \draw[->, thick] ($(CC.west)+(0, 2)$) -- ($(CC.east)+(0, -2)$);

\end{circuitikz}
\end{document}
