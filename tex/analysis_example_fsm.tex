\documentclass[border=10pt]{standalone}
\usepackage{circuitikz}

\begin{document}
\begin{circuitikz}[american]
    \ctikzset{logic ports=ieee}
    \tikzset{
        d-ff/.style={flipflop, flipflop def={
            t1=D, t6=Q, t4=\ctikztextnot{Q},
            c3=1
        }}
    }

    % Flip Flop A
    \node[d-ff] (FF) at (6, 0) {A};

    % Inputs
    \node (x) at (0, 3) {$x$};
    \node (y) at (0, 0.5) {$y$};
    \node (clk) at (4, -2) {clk};

    % XOR Gates for D input: D = A XOR x XOR y
    % Logic: (x XOR y) XOR A
    \node[xor port] (xor1) at (2, 2) {};
    \node[xor port] (xor2) at (4, 0) {};

    % Connect x, y to xor1
    \draw (x) to[short, -*] ++(1,0) |- (xor1.in 1);
    \draw (y) to[short, -*] ++(1,0) |- (xor1.in 2);

    % Connect xor1 to xor2 result
    \draw (xor1.out) |- (xor2.in 1);

    % Connect xor2 out to D input (Result of logic)
    \draw (xor2.out) -- (FF.pin 1);

    % Clock connection
    \draw (clk) |- (FF.pin 3);

    % Feedback Loop: A (Q) to xor2
    % Avoid crossing component body too messy
    \draw (FF.pin 6) -- ++(1,0) coordinate (Q_out);
    \draw (Q_out) -- ++(0, -3) -| (xor2.in 2);

    % Output Logic: z = A AND x
    \node[and port] (and1) at (8, 2) {};
    
    % Connect A to AND
    \draw (Q_out) |- (and1.in 2);

    % Connect x to AND (x is heavily far left)
    % Draw from x node logic branch
    \draw (x) -- ++(7,0) |- (and1.in 1);

    % Output z
    \draw (and1.out) -- ++(0.5,0) node[right] {$z$};

\end{circuitikz}
\end{document}
