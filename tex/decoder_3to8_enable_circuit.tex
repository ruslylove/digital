\documentclass[border=10pt]{standalone}
\usepackage[american]{circuitikz}

\begin{document}
\begin{circuitikz}
    \ctikzset{logic ports=ieee}

    % Inputs
    % E is active low, so we label it E (user spec: E=1 disables)
    \node (x) at (0, 16) {$x$};
    \node (y) at (1, 16) {$y$};
    \node (z) at (2, 16) {$z$};
    \node (e) at (3.5, 16) {$E$};

    % Vertical Rails
    \draw (x) -- (0, 3); % x rail
    \draw (y) -- (1, 3); % y rail
    \draw (z) -- (2, 3); % z rail
    

    % Inverters and Complement Rails
    \node[not port, rotate=-90, scale=0.6] (inv_x) at (0.5, 15) {};
    \draw (0, 15.5) -| (inv_x.in);
    \draw (inv_x.out) -- (0.5, 3); % x' rail

    \node[not port, rotate=-90, scale=0.6] (inv_y) at (1.5, 15) {};
    \draw (1, 15.5) -| (inv_y.in);
    \draw (inv_y.out) -- (1.5, 3); % y' rail

    \node[not port, rotate=-90, scale=0.6] (inv_z) at (2.5, 15) {};
    \draw (2, 15.5) -| (inv_z.in);
    \draw (inv_z.out) -- (2.5, 3); % z' rail

    % Enable Inverter
    % E=1 -> inv_e.out=0.
    % We want outputs=1 when E=1.
    % NAND(..., 0) = 1. So connect inv_e.out to NAND gates.
    \node[not port, rotate=-90, scale=0.6] (inv_e) at (3.5, 15) {};
    \draw (e) -- (inv_e.in);
    \draw (inv_e.out) -- (3.5, 3); % E_internal rail

    % Gates (D0 to D7)
    % Spacing: Start high, go down. 8 gates.
    % y coords roughly: 14, 12, 10, 8, 6, 4, 2, 0 (shift up a bit)
    
    \foreach \i/\val/\minterm in {
        0/000/x'y'z',
        1/001/x'y'z,
        2/010/x'yz',
        3/011/x'yz,
        4/100/xy'z',
        5/101/xy'z,
        6/110/xyz',
        7/111/xyz
    } {
        \node[nand port, number inputs=4] (D\i) at (7, 14 - \i*1.5) {};
        \draw (D\i.out) -- ++(1,0) node[right] {$D_\i$};
    }

    % Connections
    % Rails: x=0, x'=0.5, y=1, y'=1.5, z=2, z'=2.5, E=3, E'=3.5(Enable)
    
    % D0 (000): x', y', z', E'
    \draw (inv_x.out |- D0.in 1) to[short, *-] (D0.in 1);
    \draw (inv_y.out |- D0.in 2) to[short, *-] (D0.in 2);
    \draw (inv_z.out |- D0.in 3) to[short, *-] (D0.in 3);
    \draw (inv_e.out |- D0.in 4) to[short, *-] (D0.in 4);

    % D1 (001): x', y', z, E'
    \draw (inv_x.out |- D1.in 1) to[short, *-] (D1.in 1);
    \draw (inv_y.out |- D1.in 2) to[short, *-] (D1.in 2);
    \draw (z |- D1.in 3)         to[short, *-] (D1.in 3);
    \draw (inv_e.out |- D1.in 4) to[short, *-] (D1.in 4);

    % D2 (010): x', y, z', E'
    \draw (inv_x.out |- D2.in 1) to[short, *-] (D2.in 1);
    \draw (y |- D2.in 2)         to[short, *-] (D2.in 2);
    \draw (inv_z.out |- D2.in 3) to[short, *-] (D2.in 3);
    \draw (inv_e.out |- D2.in 4) to[short, *-] (D2.in 4);

    % D3 (011): x', y, z, E'
    \draw (inv_x.out |- D3.in 1) to[short, *-] (D3.in 1);
    \draw (y |- D3.in 2)         to[short, *-] (D3.in 2);
    \draw (z |- D3.in 3)         to[short, *-] (D3.in 3);
    \draw (inv_e.out |- D3.in 4) to[short, *-] (D3.in 4);

    % D4 (100): x, y', z', E'
    \draw (x |- D4.in 1)         to[short, *-] (D4.in 1);
    \draw (inv_y.out |- D4.in 2) to[short, *-] (D4.in 2);
    \draw (inv_z.out |- D4.in 3) to[short, *-] (D4.in 3);
    \draw (inv_e.out |- D4.in 4) to[short, *-] (D4.in 4);

    % D5 (101): x, y', z, E'
    \draw (x |- D5.in 1)         to[short, *-] (D5.in 1);
    \draw (inv_y.out |- D5.in 2) to[short, *-] (D5.in 2);
    \draw (z |- D5.in 3)         to[short, *-] (D5.in 3);
    \draw (inv_e.out |- D5.in 4) to[short, *-] (D5.in 4);

    % D6 (110): x, y, z', E'
    \draw (x |- D6.in 1)         to[short, *-] (D6.in 1);
    \draw (y |- D6.in 2)         to[short, *-] (D6.in 2);
    \draw (inv_z.out |- D6.in 3) to[short, *-] (D6.in 3);
    \draw (inv_e.out |- D6.in 4) to[short, *-] (D6.in 4);

    % D7 (111): x, y, z, E'
    \draw (x |- D7.in 1)         to[short, *-] (D7.in 1);
    \draw (y |- D7.in 2)         to[short, *-] (D7.in 2);
    \draw (z |- D7.in 3)         to[short, *-] (D7.in 3);
    \draw (inv_e.out |- D7.in 4) to[short, *-] (D7.in 4);

\end{circuitikz}
\end{document}
