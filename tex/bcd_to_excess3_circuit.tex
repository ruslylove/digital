\documentclass[border=10pt]{standalone}
\usepackage[american]{circuitikz}
\usetikzlibrary{shapes, arrows, positioning, calc}

\begin{document}
\begin{circuitikz}

    % Inputs
    \node (A) at (0, 6) {$A$};
    \node (B) at (0, 4) {$B$};
    \node (C) at (0, 2) {$C$};
    \node (D) at (0, 0) {$D$};


    
    % z
    \node[not port] (INV_z) at (4, -0.5) {};
    \draw (D) -- ++(1,0) |- (INV_z.in);
    \draw (INV_z.out) -- ++(0.5,0) node[right] {$z$};
    
    % y
    \node[xnor port] (XNOR_y) at (4, 1.5) {};
    \draw (C) -- ++(1.5,0) |- (XNOR_y.in 1);
    \draw (D) -- ++(2,0) |- (XNOR_y.in 2);
    \draw (XNOR_y.out) -- ++(0.5,0) node[right] {$y$};

    % T = C + D
    \node[or port] (OR_T) at (3, 3) {};
    \draw (C) -- ++(1.5,0) |- (OR_T.in 2);
    %\draw (D) -- ++(1,0) |- (OR_T.in 2); % This is messy. Let's assume C and D available.
    % Actually, let's tap from the lines that go to XNOR?
    % Hard to draw perfect shared lines without 'short'.
    \draw (D) -- ++(1,0) |- (OR_T.in 2); % D is lowest
    
    % Correction: OR_T takes C and D.
    % Re-route:
    \draw (C) -- ++(1.2,0) |- (OR_T.in 1);
    
    % x = B xor T
    \node[xor port] (XOR_x) at (6, 4) {};
    \draw (B) -- ++(3,0) |- (XOR_x.in 1);
    \draw (OR_T.out) -- ++(0.5,0) |- (XOR_x.in 2);
    \draw (XOR_x.out) -- ++(0.5,0) node[right] {$x$};
    
    % w = A + (B . T)
    \node[and port] (AND_BT) at (6, 5.5) {};
    \draw (B) -- ++(2.8,0) |- (AND_BT.in 2);
    \draw (OR_T.out) -- ++(0.5,0) |- (AND_BT.in 1); % Shared T
    
    \node[or port] (OR_w) at (8, 6) {};
    \draw (A) -- ++(6,0) |- (OR_w.in 1);
    \draw (AND_BT.out) |- (OR_w.in 2);
    \draw (OR_w.out) -- ++(0.5,0) node[right] {$w$};

\end{circuitikz}
\end{document}
