\documentclass[border=10pt]{standalone}
\usepackage{tikz}
\usepackage{circuitikz}
\usetikzlibrary{positioning, calc, decorations.markings}

\tikzset{register/.style={muxdemux, muxdemux def={Lh=2, Rh=2, NL=2, NB=1, NT=1,w=4,NR=0},
                            muxdemux label={L1=Load, L2=Clock, R1=right1, R2=right,T1=$D_{7..0}$,B1=$Q_{7..0}$,},},
                 }

\tikzset{adder/.style={muxdemux, muxdemux def={Lh=1, Rh=1,NL=0, NB=1, NT=2,w=2,NR=0}}}

\begin{document}
\begin{circuitikz}[
    font=\sffamily,
    arrow/.style={-Latex, line width=1pt},
    label_text/.style={font=\footnotesize},
    bus/.style={-Latex, line width=1.5pt,
        postaction={decorate},
        decoration={markings, mark=at position 0.5 with {
            \draw[thick,-] (-2pt,-3pt) -- (2pt,3pt);
            \node[above left=0.5pt] {\footnotesize $8$};
        }}},
    control_signal/.style={-Latex, line width=0.5pt},
]

    % Components
    % Sources B and C at the top
    \node[register] at (-2,3.5) (B) {$B$};
    \node[register] at (2,3.5) (C) {$C$};
    
    % Adder in the middle
    \node[adder, box only, anchor=center] at (0,1) (Add) {+};
    
    % Destination A at the bottom
    \node[register] at (0,-0.8) (A) {$A$};

    % Connections
    % B (Bottom) to Adder (Top Pin 1)
    % Use coordinate calculations for clean Manhattan lines
    \draw[bus] (B.bbpin 1) -- ++(0,-1) -| (Add.btpin 1);
    
    % C (Bottom) to Adder (Top Pin 2)
    \draw[bus] (C.bbpin 1) -- ++(0,-1) -| (Add.btpin 2);
    
    % Adder (Bottom) to A (Top)
    \draw[bus] (Add.bbpin 1) -- (A.btpin 1);


    % Control Signals
    % Common Clock Rail
    \draw[control_signal] (-4, -1.5) node[below, label_text] (ClkLabel) {$Clock$} |- (B.blpin 2);
    
    % Clock Connections
    \draw[control_signal] (-4, 0 |- A.lpin 2) to[short,*-] (A.blpin 2);
   % \draw[arrow] (-6, 0 |- B.lpin 2) to[short,*-] (B.blpin 2);
    % Clock to C (route above B)
    \draw[control_signal] (-4, 2.5) to[short,*-] ++(4,0) |- (C.blpin 2);

    % Load Signals
    \draw[control_signal] (-5, 0 |- A.lpin 1) node[left, label_text] {$ALoad$} -- (A.blpin 1);
    \draw[control_signal] (-5, 0 |- B.lpin 1) node[left, label_text] {$BLoad$} -- (B.blpin 1);
    \draw[control_signal] (-5, 4.5) node[left, label_text] {$CLoad$} -- ++(5,0) |- (C.blpin 1);

\end{circuitikz}
\end{document}
