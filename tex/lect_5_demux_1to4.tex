\documentclass[border=10pt]{standalone}
\usepackage[american]{circuitikz}
\usetikzlibrary{calc}

\begin{document}
\begin{circuitikz}
    \ctikzset{logic ports=ieee, scale=0.7}

    % Define nodes for inputs at the top
    \node (data) at (0, 4) {Input ($D$)};
    \node (sel1) at (-3, 4) {$S_1$};
    \node (sel0) at (-6, 4) {$S_0$};

    % Vertical Rails
    % Data Rail
    \draw (data) -- (0, -10);
    
    % S1 Rail and Inverter
    %\draw (sel1) -- (0.5, 2);
    \draw (sel1) -- ++(0, -1) node[circ] (s1_tap) {};
    \draw (s1_tap) -- ++(0,-0.5) node[not port, scale=0.5, rotate=-90, anchor=in] (not1) {}; 
    \draw (s1_tap) -- ++ (-1,0) coordinate (s1) {};
    \draw (s1) -- ++ (0,-13);
    \draw (not1.out) -- (-3, -10); % S1' inverse
    
    % S0 Rail and Inverter
    \draw (sel0) -- ++(0, -1) node[circ] (s0_tap) {};
    \draw (s0_tap) -- ++(0,-0.5) node[not port, rotate=-90, scale=0.5,anchor=in] (not0) {}; 
    \draw (s0_tap) -- ++ (-1,0) coordinate (s0) {};
    \draw (s0) -- ++ (0,-13);
    \draw (not0.out) -- (-6, -10);
    % Labels for rails (optional, skipping for cleanliness)

    % AND Gates
    % Y0 = D . S1' . S0' (Inputs: D, S1', S0')
    \node[and port, number inputs=3] (and0) at (5, 0) {};
    \draw (and0.in 1) to[short, -*] (and0.in 1 -|data);
    \draw (and0.in 2) to[short, -*] (and0.in 2 -|s0_tap);
    \draw (and0.in 3) to[short, -*] (and0.in 3 -|s1_tap);

    % Y1 = D . S1' . S0  (Inputs: D, S1', S0)
    \node[and port, draw, number inputs=3] (and1) at (5, -3) {};
    \draw (and1.in 1) to[short, -*] (and1.in 1 -|data);
    \draw (and1.in 2) to[short, -*] (and1.in 2 -|s0_tap);
    \draw (and1.in 3) to[short, -*] (and1.in 3 -|s1);
    % Y2 = D . S1 . S0'  (Inputs: D, S1, S0')
    \node[and port, draw, number inputs=3] (and2) at (5, -6) {};
    \draw (and2.in 1) to[short, -*] (and2.in 1 -|data);
    \draw (and2.in 2) to[short, -*] (and2.in 2 -|s0_tap);
    \draw (and2.in 3) to[short, -*] (and2.in 3 -|s1);
    % Y3 = D . S1 . S0   (Inputs: D, S1, S0)
    \node[and port, draw, number inputs=3] (and3) at (5, -9) {};
    \draw (and3.in 1) to[short, -*] (and3.in 1 -|data);
    \draw (and3.in 2) to[short, -*] (and3.in 2 -|s0);
    \draw (and3.in 3) to[short, -*] (and3.in 3 -|s1);

    % Outputs
    \draw (and0.out) -- ++(0.5,0) node[right] {$Y_0$};
    \draw (and1.out) -- ++(0.5,0) node[right] {$Y_1$};
    \draw (and2.out) -- ++(0.5,0) node[right] {$Y_2$};
    \draw (and3.out) -- ++(0.5,0) node[right] {$Y_3$};

    % Connections
    
    

\end{circuitikz}
\end{document}
