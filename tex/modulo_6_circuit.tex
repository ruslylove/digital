\documentclass[border=10pt]{standalone}
\usepackage{circuitikz}

\begin{document}
\begin{circuitikz}[american, scale=1.0]
    \ctikzset{logic ports=ieee} 
    % Flip-Flops
    \tikzset{d-ff/.style={flipflop, flipflop def={t1=D, t6=Q, t4=\ctikztextnot{Q}, c3=1}}}
    
    % Nodes
    \node (C) at (-2, 8) {$C$};
    \node (clk) at (6, -1) {CLK};
    
    % FFs
    \node[d-ff] (FF0) at (6, 1) {$Q_0$};
    \node[d-ff] (FF1) at (6, 5) {$Q_1$};
    \node[d-ff] (FF2) at (6, 9) {$Q_2$};
    
    % Clock distribution
    \draw (clk) -| (FF0.pin 3);
    \draw (clk) -| (FF1.pin 3);
    \draw (clk) -| (FF2.pin 3);
    
    % Logic D0 = Q0 XOR C
    \node[xor port] (xor_d0) at (2, 1) {};
    \draw (FF0.pin 6) -- ++(0.5,0) coordinate(Q0_out) |- (xor_d0.in 1);
    \draw (C) |- (xor_d0.in 2);
    \draw (xor_d0.out) -- (FF0.pin 1);
    
    % Logic D2 = Q2 Q1' + Q2 Q0 C' + Q1 Q0 C
    % Terms
    \node[and port] (term1_d2) at (0, 11) {}; % Q2 Q1'
    \node[and port, number inputs=3] (term2_d2) at (0, 9.5) {}; % Q2 Q0 C'
    \node[and port, number inputs=3] (term3_d2) at (0, 8) {}; % Q1 Q0 C
    \node[or port, number inputs=3] (or_d2) at (3, 9) {};

    % Connections D2
    % Q2 Q1'
    \draw (FF2.pin 6) -- ++(0.5, 0) coordinate(Q2_out) |- (term1_d2.in 1);
    \draw (FF1.pin 4) -- ++(0.5, -0.5) -- ++(-1, 0) coordinate(Q1_bar_out) |- (term1_d2.in 2);
    
    % Q2 Q0 C'
    \draw (Q2_out) |- (term2_d2.in 1);
    \draw (Q0_out) |- (term2_d2.in 2);
    \node[not port, scale=0.6] (inv_C1) at (-1, 8) {};
    \draw (C) |- (inv_C1.in);
    \draw (inv_C1.out) |- (term2_d2.in 3); % C'
    
    % Q1 Q0 C
    \draw (FF1.pin 6) -- ++(0.5, 0) coordinate(Q1_out) |- (term3_d2.in 1);
    \draw (Q0_out) |- (term3_d2.in 2);
    \draw (C) |- (term3_d2.in 3);
    
    % OR for D2
    \draw (term1_d2.out) -- (or_d2.in 1);
    \draw (term2_d2.out) -- (or_d2.in 2);
    \draw (term3_d2.out) -- (or_d2.in 3);
    \draw (or_d2.out) -- (FF2.pin 1);
    
    % Logic D1 = Q1 Q0' + Q1 Q0 C' + Q2' Q1' Q0 C
    % Terms
    \node[and port] (term1_d1) at (0, 6) {}; % Q1 Q0'
    \node[and port, number inputs=3] (term2_d1) at (0, 4.5) {}; % Q1 Q0 C'
    \node[and port, number inputs=4] (term3_d1) at (0, 3) {}; % Q2' Q1' Q0 C
    \node[or port, number inputs=3] (or_d1) at (3, 5) {};
    
    % Connections D1
    % Q1 Q0'
    \draw (Q1_out) |- (term1_d1.in 1);
    \draw (FF0.pin 4) -- ++(0.5, -0.5) -- ++(-1, 0) coordinate(Q0_bar_out) |- (term1_d1.in 2);
    
    % Q1 Q0 C'
    \draw (Q1_out) |- (term2_d1.in 1);
    \draw (Q0_out) |- (term2_d1.in 2);
    \draw (inv_C1.out) |- (term2_d1.in 3); % Reuse C'
    
    % Q2' Q1' Q0 C
    \draw (FF2.pin 4) |- (term3_d1.in 1); % Q2'
    \draw (Q1_bar_out) |- (term3_d1.in 2); % Q1'
    \draw (Q0_out) |- (term3_d1.in 3); % Q0
    \draw (C) |- (term3_d1.in 4); % C
    
    % OR for D1
    \draw (term1_d1.out) -- (or_d1.in 1);
    \draw (term2_d1.out) -- (or_d1.in 2);
    \draw (term3_d1.out) -- (or_d1.in 3);
    \draw (or_d1.out) -- (FF1.pin 1);
    
    % Output Y = Q2 Q0
    \node[and port] (and_y) at (8, 5) {};
    \draw (Q2_out) -| (and_y.in 1);
    \draw (Q0_out) -| (and_y.in 2);
    \draw (and_y.out) -- ++(0.5,0) node[right] {$Y$};

\end{circuitikz}
\end{document}
