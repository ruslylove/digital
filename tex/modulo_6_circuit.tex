\documentclass[border=10pt]{standalone}
\usepackage{circuitikz}

\usetikzlibrary{backgrounds}
\begin{document}
\begin{circuitikz}[american, scale=1.0]
    \ctikzset{logic ports=ieee}
    \ctikzset{logic ports/fill=white} 
    % Flip-Flops
    \tikzset{d-ff/.style={flipflop, flipflop def={t1=D, t6=Q, t4=\ctikztextnot{Q}, c3=1}}}
    
    % Q-Bus Nodes
    \coordinate (Q2_bus) at (-8.5, 15);
    \coordinate (Q2_bar_bus) at (-7.5, 15);
    \coordinate (Q1_bus) at (-6.5, 15);
    \coordinate (Q1_bar_bus) at (-5.5, 15);
    \coordinate (Q0_bus) at (-4.5, 15);
    \coordinate (Q0_bar_bus) at (-3.5, 15);
    \coordinate (C_bus) at (-2.5, 15);
    \coordinate (C_bar_bus) at (-1.5, 15);
    
    % Draw Buses
    \draw (Q2_bus) node[above] {$(Q_2)$} -- (-8.5, -3);
    \draw (Q2_bar_bus) node[above] {$(Q_2')$} -- (-7.5, -3);
    \draw (Q1_bus) node[above] {$(Q_1)$} -- (-6.5, -3);
    \draw (Q1_bar_bus) node[above] {$(Q_1')$} -- (-5.5, -3);
    \draw (Q0_bus) node[above] {$(Q_0)$} -- (-4.5, -3);
    \draw (Q0_bar_bus) node[above] {$(Q_0')$} -- (-3.5, -3);
    \draw (C_bus) node[above] {$C$} -- (-2.5, -3);
   % \draw (C_bar_bus) node[above] {$C'$} -- (-1.5, -1);
    
    % Inverter for C -> C'
    \path (C_bus) -- ++(0,-0.5) coordinate (c_tap);
    \node[circ] (tmp) at (c_tap) {};
    \node[not port, scale=0.5, anchor=in] (c_inv) at (c_tap) {};
    \draw (c_inv.out) -- (c_inv.out -| C_bar_bus) -- (-1.5,-3);
   % \draw (-2, 12.5) -- (c_inv.in);
   % \draw (c_inv.out) -- (-1.5, 12.5);
    
    \node[left] (clk) at (3, -3) {CLK};
    
    % FFs
    \node[d-ff] (FF0) at (6, 1) {$Q_0$};
    \node[d-ff] (FF1) at (6, 5) {$Q_1$};
    \node[d-ff] (FF2) at (6, 9) {$Q_2$};
    
    % Clock distribution & Q Routing
    \begin{scope}[on background layer]
  %      \draw (clk) -| (FF0.pin 3);
  %      \draw (clk) -| (FF1.pin 3);
        \draw (clk) -| (FF2.pin 3);
        
        % Route Q to Bus
   %     \draw (FF0.pin 6) -- ++(1,0) -- ++(0, 5) -- ++(-11, 0) -- (-3, 6 |- Q0_bus); % Route Q0 up and over to -3
        % Wait, routing up and over might be messy. Let's route horizontally through the gap if possible or go up/down.
        % Actually, simpler: Route directly left if 'background' works well.
        % FF inputs (D) are on pin 1 (left). FF outputs (Q) are on pin 6 (right).
        % Drawing from right side back to left side requires going around.
        % Let's go UP from each Q, then LEFT above everything (y=12.5), then DOWN to the bus.
        
        % Q2 Route
    %    \draw (FF2.pin 6) -- ++(0.5,0) -- ++(0, 0.5) -- (-5, 9.5) -- (Q2_bus |- FF2.pin 6); % Just connect to the vertical line? No, the vertical line is the bus.
        % Let's just define the bus as the source.
        % We need to drive the bus FROM the FFs.
        % Route Q2
    %    \draw (FF2.pin 6) -- ++(0.5,0) -- ++(0, 2) -- (-5, 11) -- (-5, 9); % Connect to bus line at -5
        % Route Q1
    %    \draw (FF1.pin 6) -- ++(0.5,0) -- ++(0, 6) -- (-4, 11) -- (-4, 5); % Connect to bus line at -4
        % Route Q0
    %    \draw (FF0.pin 6) -- ++(0.5,0) -- ++(0, 10) -- (-3, 11) -- (-3, 1); % Connect to bus line at -3
        
        % Re-thinking routing: The FFs are stacked at x=6, y=1,5,9.
        % The logic is to the left (x=0..3).
        % The buses are further left (x=-5..-2).
        % So logic inputs take from buses. Logic outputs go to FF inputs (pin 1, left side).
        % FF outputs (pin 6, right side) need to go to buses (far left).
        % So loop back: Q -> Right -> Up/Down -> Left -> Bus
    \end{scope}
    
    % Better Routing Strategy:
    % Q2 (6,9) -> (7,9) -> (7,12) -> (-5,12) -> (-5, -1)
    % Q1 (6,5) -> (6.5,5) -> (6.5,12.2) -> (-4,12.2) -> (-4, -1)
    % Q0 (6,1) -> (6.2,1) -> (6.2,12.4) -> (-3,12.4) -> (-3, -1)
    
    \begin{scope}[on background layer]
        % Clock
      %  \draw (clk) -| (FF0.pin 3);
      %  \draw (clk) -| (FF1.pin 3);
        \draw (clk) -| (FF2.pin 3);
        
        % Q2 Feedback
    \draw (FF2.pin 6) -- ++(0, 2) coordinate (q2);
    \draw (q2) to[short,-*] (q2 -| Q2_bus);
        % Q1 Feedback
    \draw (FF1.pin 6) -- ++(1,0) -- ++(0, 6.6) coordinate (q1);
    \draw (q1) to[short,-*] (q1 -| Q1_bus);
        % Q0 Feedback
    \draw (FF0.pin 6) -- ++(2,0) -- ++(0, 11.2) coordinate (q0);
    \draw (q0) to[short,-*] (q0 -| Q0_bus);
        
        % Q2' Feedback
    \draw (FF2.pin 4) -- ++(0.5,0) -- ++(0, 4) coordinate (q2_bar);
    \draw (q2_bar) to[short,-*] (q2_bar -| Q2_bar_bus);
        % Q1' Feedback
    \draw (FF1.pin 4) -- ++(1.5,0) -- ++(0, 8.6) coordinate (q1_bar);
    \draw (q1_bar) to[short,-*] (q1_bar -| Q1_bar_bus);
        % Q0' Feedback
    \draw (FF0.pin 4) -- ++(2.5,0) -- ++(0, 13.2) coordinate (q0_bar);
    \draw (q0_bar) to[short,-*] (q0_bar -| Q0_bar_bus);
    \end{scope}
    
    % Logic D0 = Q0 XOR C (D0 = Q0 + C)
    % D0 = Q0 XOR C.
    \node[xor port] (xor_d0) at (2, 1) {};
    \begin{scope}[on background layer]
        \draw (Q0_bus |- xor_d0.in 1) to[short,*-] (xor_d0.in 1); % Q0
        \draw (C_bus |- xor_d0.in 2) to[short,*-] (xor_d0.in 2); % C
        \draw (xor_d0.out) |- (FF0.pin 1);
    \end{scope}
    
    % Logic D2 = Q2 Q1' + Q2 Q0 C' + Q1 Q0 C
    % Terms
    \node[and port] (term1_d2) at (0, 11) {}; % Q2 Q0'
    \node[and port] (term2_d2) at (0, 9.5) {}; % Q2 C'
    \node[and port, number inputs=3] (term3_d2) at (0, 8) {}; % Q1 Q0 C
    \node[or port, number inputs=3] (or_d2) at ($(term2_d2) + (3, 0)$) {};

    % Connections D2
    \begin{scope}[on background layer]
        % Q2 Q0'
        % Input 1: Q2 from bus
        \draw (Q2_bus |- term1_d2.in 1) to[short,*-] (term1_d2.in 1);
        % Input 2: Q0' (from bus)
        \draw (Q0_bar_bus |- term1_d2.in 2) to[short,*-] (term1_d2.in 2);
        
        % Q2 C'
        % Input 1: Q2
        \draw (Q2_bus |- term2_d2.in 1) to[short,*-] (term2_d2.in 1);
        % Input 2: C' (from bus)
        \draw (C_bar_bus |- term2_d2.in 2) to[short,*-] (term2_d2.in 2);
        
        % Q1 Q0 C
        \draw (Q1_bus |- term3_d2.in 1) to[short,*-] (term3_d2.in 1); % Q1
        \draw (Q0_bus |- term3_d2.in 2) to[short,*-] (term3_d2.in 2); % Q0
        \draw (C_bus |- term3_d2.in 3) to[short,*-] (term3_d2.in 3); % C
        
        % OR
        \draw (term1_d2.out) |- (or_d2.in 1);
        \draw (term2_d2.out) |- (or_d2.in 2);
        \draw (term3_d2.out) |- (or_d2.in 3);
        \draw (or_d2.out) |- (FF2.pin 1);
    \end{scope}
    
    % Logic D1 = Q1 Q0' + Q1 Q0 C' + Q2' Q1' Q0 C
    % Terms
    \node[and port] (term1_d1) at (0, 6) {}; % Q1 Q0'
    \node[and port] (term2_d1) at (0, 4.5) {}; % Q1 C'
    \node[and port, number inputs=4] (term3_d1) at (0, 3) {}; % Q2' Q1' Q0 C
    \node[or port, number inputs=3] (or_d1) at ($(term2_d1) + (3, 0)$) {};
    
    % Connections D1
    \begin{scope}[on background layer]
        % Q1 Q0'
        \draw (Q1_bus |- term1_d1.in 1) to[short,*-] (term1_d1.in 1); % Q1
        % Q0'
        \draw (Q0_bar_bus |- term1_d1.in 2) to[short,*-] (term1_d1.in 2);
        
        % Q1 C'
        \draw (Q1_bus |- term2_d1.in 1) to[short,*-] (term2_d1.in 1); % Q1
        % C'
        \draw (C_bar_bus |- term2_d1.in 2) to[short,*-] (term2_d1.in 2);
        
        % Q2' Q1' Q0 C
        % Q2'
        \draw (Q2_bar_bus |- term3_d1.in 1) to[short,*-] (term3_d1.in 1);
        % Q1'
        \draw (Q1_bar_bus |- term3_d1.in 2) to[short,*-] (term3_d1.in 2);
        % Q0
        \draw (Q0_bus |- term3_d1.in 3) to[short,*-] (term3_d1.in 3);
        % C
        \draw (C_bus |- term3_d1.in 4) to[short,*-] (term3_d1.in 4);
        
        % OR
        \draw (term1_d1.out) |- (or_d1.in 1);
        \draw (term2_d1.out) |- (or_d1.in 2);
        \draw (term3_d1.out) |- (or_d1.in 3);
        \draw (or_d1.out) |- (FF1.pin 1);
    \end{scope}
    
    % Output Y = Q2 Q1' Q0
    \node[and port, number inputs=3] (and_y) at (9, -1.5) {};
    \begin{scope}[on background layer]
        \draw (Q2_bus |- and_y.in 1) to[short,*-] (and_y.in 1); % Q2
        % Q1'
        \draw (Q1_bar_bus |- and_y.in 2) to[short,*-] (and_y.in 2);
        % Q0
        \draw (Q0_bus |- and_y.in 3) to[short,*-] (and_y.in 3); % Q0
        \draw (and_y.out) -- ++(0.5,0) node[right] {$Y$};
    \end{scope}

\end{circuitikz}
\end{document}
