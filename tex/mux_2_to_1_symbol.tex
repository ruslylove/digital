\documentclass[border=10pt]{standalone}
\usepackage{circuitikz}

\begin{document}

\begin{circuitikz}[thick]
    \node [muxdemux, muxdemux def={Lh=4, NL=2, Rh=3, NR=1, NB=1, w=2, square pins=1}] (mux) at (0,0) {};
    
    % Input labels
    \node [left] at (mux.lpin 1) {$x_1$};
    \node [left] at (mux.lpin 2) {$x_2$};

    % Pin numbers
    \node [right][xshift=8pt] at (mux.lpin 1) {0};
    \node [right][xshift=8pt] at (mux.lpin 2) {1};
    
    % Output label
    \node [right] at (mux.rpin 1) {$f$};
    
    % Select label
    \node [below] at (mux.bpin 1) {$s$};
    
    % Internal pin numbers (optional, matching previous style if needed, but standard symbol usually doesn't show them inside like that)
    %\node [right, font=\tiny] at (mux.lpin 1) {0};
    %\node [right, font=\tiny] at (mux.lpin 2) {1};
    
\end{circuitikz}

\end{document}