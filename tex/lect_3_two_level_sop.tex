\documentclass[border=10pt]{standalone}
\usepackage{circuitikz}

\ctikzset{
logic ports=ieee,
logic ports/scale=0.7,
}

\begin{document}
\begin{circuitikz}[]

    % 1. Place the Gates
    % Top AND Gate (For A and B)
    \draw (0, 3) node[and port] (AND1) {};
    
    % Bottom AND Gate (For C and D)
    \draw (0, 0) node[and port] (AND2) {};

    % OR Gate (Summing everything)
    % We place this to the right (x=4) and vertically centered (y=1.5).
    % KEY PARAMETER: 'number inputs=3' creates 3 input anchors.
    \draw (3, 1.5) node[or port, number inputs=3] (OR1) {};


    % 2. Draw AND Gate Inputs (A, B, C, D)
    \draw (AND1.in 1) -- ++(-1, 0) node[left] {$A$};
    \draw (AND1.in 2) -- ++(-1, 0) node[left] {$B$};
    
    \draw (AND2.in 1) -- ++(-1, 0) node[left] {$C$};
    \draw (AND2.in 2) -- ++(-1, 0) node[left] {$D$};


    % 3. Connect Gates to OR Gate
    % Connect Top AND to OR Top Input (in 1)
    % We use 'to[short]' or simple '--' lines.
    % The syntax '|- (OR1.in 1)' ensures a nice right-angle turn.
    \draw (AND1.out) -| (OR1.in 1);
    
    % Connect Bottom AND to OR Bottom Input (in 3)
    \draw (AND2.out) -| (OR1.in 3);


    % 4. Draw Input E (The Direct Connection)
    % We connect directly to the Middle Input (in 2) of the OR gate.
    % We draw the line backwards to the left so the label aligns with A-D.
    \draw (OR1.in 2) -- ++(-4, 0) node[left] {$E$};


    % 5. Output
    \draw (OR1.out) -- ++(1, 0) node[right] {$F = AB + CD + E$};

\end{circuitikz}
\end{document}