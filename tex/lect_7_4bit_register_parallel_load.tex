\documentclass[border=10pt]{standalone}
\usepackage{circuitikz}
\usetikzlibrary{calc}

\begin{document}
\begin{circuitikz}[american, scale=1.0]
    \ctikzset{logic ports=ieee}
    \ctikzset{flipflops/scale=0.8}
    
    % Common Signals
    \node (clk_in) at (-4, -1.5) {CLK};
    \node (clr_in) at (-4, -2) {Clear};
    \node (load_in) at (-4, 2.5) {Load};
    
    % Positioning
    \def\ffdist{5.5}
    
    \foreach \i in {3,2,1,0} {
        \pgfmathsetmacro{\xpos}{(3-\i)*\ffdist}
        
        % Flip-Flop
        \node[flipflop D] (FF\i) at (\xpos, 0) {};
        \node[above] at (FF\i.n) {$D_\i$};
        \draw (FF\i.pin 6) -- ++(0.5, 0) node[right] {$A_\i$};
        
        
    
        
        % MUX using muxdemux node
        % muxdemux shape: 
        % anchors: .lpin 1..n, .rpin 1..n, .tpin 1..n, .bpin 1..n, .select n?
        % For 2-to-1: muxdemux, muxdemux def={Lh=4, Rh=3, w=2, ...}?
        % Or use standard "muxdemux" with "select pins=1, muxtdemux pins=2"?
        % Syntax: \node [muxdemux, muxdemux def={NL=2, NR=1, NB=1, NT=1}] ...
        % Let's try simpler standard usage if defaults work, or configure.
        % "muxdemux" is a shape. 
        % Default is often 2 inputs on left, 1 output on right.
        
        % Let's try basic configuration.
        % NL=2 (2 left inputs), NR=1 (1 right output), NB=1 (1 bottom select - usually select is on top/bottom)
        % But "muxdemux" shape is very flexible. 
        % Let's use: muxdemux, muxdemux def={Lh=4, NL=2, Rh=2, NR=1, NB=0, NT=1, w=1.5, inset_w=0.5, inset_Lh=1, inset_Rh=0}
        % Actually, a simple `muxdemux` might default to something useful or we specify pins.
        % Let's use: `muxdemux, muxdemux def={NL=2, NR=1, NT=1}` (Select on Top)
        
        \coordinate (mux_pos_\i) at ($(FF\i.pin 1) + (-0.3, 0)$);
        
        \node [muxdemux, 
               muxdemux def={NL=2, NR=1, NT=1, NB=0, w=1, Lh=3, Rh=2},
               rotate=0, anchor=rpin 1] (MUX\i) at (mux_pos_\i) {};
        
        % Connect MUX Output (rpin 1) to D input (already aligned by anchor?)
        % If anchored by rpin 1, then MUX\i.rpin 1 is at mux_pos_\i.
        \draw (MUX\i.rpin 1) -- (FF\i.pin 1);
        
        % MUX Input 1 (lpin 2 - Top? Or lpin 1 Top? Standard is 1 top, 2 bottom)
        % Let's check pin numbering. Usually 1 is top.
        \draw (MUX\i.lpin 1) -- ++(-0.5, 0) node[label={left:$I_\i$}] {};
        \node[right, font=\small][xshift=+0.6em] at (MUX\i.lpin 1) {1};

        % MUX Input 0 (lpin 2 - Bottom)
        \draw (MUX\i.lpin 2) -- ++(-0.2, 0) coordinate (fb_start);
        \node[right, font=\small][xshift=+0.6em] at (MUX\i.lpin 2) {0};
        
        % Feedback Path from A_i
        \coordinate (out_pt) at ($(FF\i.pin 6) + (0.2, 0)$);
        \draw (out_pt) ++(0.1, 0) coordinate (tap);
        \draw (tap) to[short, *-] ++(0, +1.2) coordinate (fb_turn1);
        \draw (fb_turn1) -- (fb_turn1 -| fb_start) coordinate (fb_turn2);
        \draw (fb_turn2) -- (fb_start) -- (MUX\i.lpin 2);
        
        % Load Signal Connection (tpin 1 - Top Select)
        \ifnum\i=0
            % skip last mux
        \else
        \draw (MUX\i.tpin 1) -- (MUX\i.tpin 1 |- load_in) node[circ] {};
            % Clear connection
        \draw (FF\i.s) -- ++(0,-0.2) coordinate (clr_pt);
        \draw (clr_pt) -- (clr_pt |- clr_in) node[circ] {};
        \draw (FF\i.pin 3) -- (FF\i.pin 3 |- clk_in) node[circ] {}; 
        \fi
    }
    
    % Finish Rails
    \draw (clk_in) -| (FF0.pin 3);
    \draw (clr_in) -| (FF0.s);
    \draw (load_in) -| (MUX0.tpin 1);
    
\end{circuitikz}
\end{document}
