\documentclass[border=10pt]{standalone}
\ifdefined\pdfoutput
    \ifnum\pdfoutput>0
        % in PDF mode, do not use the dvisvgm driver
    \else
        \def\pgfsysdriver{pgfsys-dvisvgm.def}
    \fi
\else
    \def\pgfsysdriver{pgfsys-dvisvgm.def}
\fi
\usepackage[siunitx]{circuitikz}
\ctikzset{logic ports=ieee}

\tikzset{d-ff/.style={flipflop, flipflop def={
t1=D, t3=C, t4={\ctikztextnot{Q}},
t6=Q}},
}



\begin{document}
  \begin{circuitikz}
    % Master Latch
    \node[d-ff] (Master) at (0,0) {Master};
    % Slave Latch
    \node[d-ff] (Slave) at (4,0) {Slave};
    
    % Wiring
    % Input D
    \draw (Master.pin 1) -- ++(-1, 0) node[left] {D};
    
    % Clock Input common point
    \draw (Master.pin 3) -- ++(-0.5, 0) coordinate (clk_split);
    \draw (clk_split) -- ++(-0.5, 0) node[left] {CLK};
    
    % Inverter logic for Slave
    \draw (clk_split) -- ++(0, -1) coordinate (turn);
    \draw (turn) -- ++(3, 0) node[not port, anchor=in, scale=0.8] (inv) {};
    % From INV out to Slave E (pin 3 of Slave)
    \draw (inv.out) -| (Slave.pin 3);
    
    % Interconnect Master Q -> Slave D
    \draw (Master.pin 6) -- (Slave.pin 1);
    
    % Outputs
    \draw (Slave.pin 6) -- ++(0.5, 0) node[right] {$Q$};
    \draw (Slave.pin 4) -- ++(0.5, 0) node[right] {$\bar{Q}$};
  \end{circuitikz}
\end{document}
