\documentclass[border=10pt]{standalone}
\ifdefined\pdfoutput
    \ifnum\pdfoutput>0
        % PDF mode
    \else
        \def\pgfsysdriver{pgfsys-dvisvgm.def}
    \fi
\else
    \def\pgfsysdriver{pgfsys-dvisvgm.def}
\fi
\usepackage{circuitikz}
\usetikzlibrary{positioning}

\begin{document}
\begin{circuitikz}[american]
    \ctikzset{logic ports=ieee}
    
    % Define styles
    \tikzset{
        latch-base/.style={flipflop, flipflop def={t6=Q, t4=\ctikztextnot{Q}}},
        sr-latch/.style={latch-base, flipflop def={t1=S, t3=R}},
        % Note: n1=1, n3=1 is a guess for input negation. If fails, will fix.
        srb-latch/.style={latch-base, flipflop def={t1=S, t3=R, n1=1, n3=1}}, 
        d-latch/.style={latch-base, flipflop def={t1=D, t3=C, c3=0}}, % c3=0 explicit no wedge
    }

    % === SR Latch Symbol ===
    \node[sr-latch] (sr) at (0,0) {};
    \node[above=0.1cm of sr] {SR Latch};
    % Wires
    \draw (sr.pin 1) -- ++(-0.5,0);
    \draw (sr.pin 3) -- ++(-0.5,0);
    \draw (sr.pin 6) -- ++(0.5,0);
    \draw (sr.pin 4) -- ++(0.5,0);

    % === S'R' Latch Symbol (Active Low) ===
    \node[srb-latch] (srb) at (5,0) {};
    \node[above=0.1cm of srb] {$\bar{S}\bar{R}$ Latch};
    % Wires
    \draw (srb.pin 1) -- ++(-0.5,0);
    \draw (srb.pin 3) -- ++(-0.5,0);
    \draw (srb.pin 6) -- ++(0.5,0);
    \draw (srb.pin 4) -- ++(0.5,0);


    % === D Latch Symbol ===
    \node[d-latch] (dl) at (10,0) {};
    \node[above=0.1cm of dl] {D Latch};
    % Wires
    \draw (dl.pin 1) -- ++(-0.5,0);
    \draw (dl.pin 3) -- ++(-0.5,0);
    \draw (dl.pin 6) -- ++(0.5,0);
    \draw (dl.pin 4) -- ++(0.5,0);

\end{circuitikz}
\end{document}
