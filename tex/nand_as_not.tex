\documentclass[border=10pt]{standalone}
\usepackage{circuitikz}

\ctikzset{
logic ports=ieee,
logic ports/scale=0.7,
}

\begin{document}

\begin{circuitikz}[] % Use standard American logic gate symbols

    % 1. Place the NAND gate
    % We place it at coordinates (2,0) and name it 'NAND1'
    \draw (2,0) node[nand port] (NAND1) {};

    % 2. Draw the Input and Split Wire
    % Start a wire from (-1, 0) labeled 'A'
    % Draw it to coordinate (0.5, 0), which is where we will split it.
    % Add a solder dot '*' at the end of this segment.
    \draw (-1,0) node[left] {$A$} -- (0.5,0) node[circ] (splitPoint) {};

    % 3. Connect the Split Point to the Gate Inputs
    % Connect to input 1 (top) using vertical-then-horizontal path |-
    \draw (splitPoint) |- (NAND1.in 1);
    % Connect to input 2 (bottom)
    \draw (splitPoint) |- (NAND1.in 2);

    % 4. Draw the Output
    % Draw a wire from the gate output to the right and label it.
    \draw (NAND1.out) -- ++(1,0) node[right] {$Y = \overline{A}$};

\end{circuitikz}

\end{document}