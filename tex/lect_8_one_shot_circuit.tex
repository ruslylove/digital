\documentclass[border=10pt]{standalone}
\usepackage{circuitikz}

\begin{document}
\begin{circuitikz}[american, scale=1.0]
    \ctikzset{logic ports=ieee} 
    % Flip-Flops
    \tikzset{d-ff/.style={flipflop, flipflop def={t1=D, t6=Q, t4=\ctikztextnot{Q}, c3=1}}}
    
    % Nodes
 %   \node (B) at (-2, 6) {$B$};
 %   \node (clk) at (6, -1) {CLK};
    
    % FFs
    \node[d-ff] (FF0) at (6, 1) {$Q_0$};
    \node[d-ff] (FF1) at (6, 5) {$Q_1$};
    
    % Clock
 %   \draw (clk) -| (FF0.pin 3);
 %   \draw (clk) -| (FF1.pin 3);
    
    % Logic D0 = S = Q1' Q0' B
    \draw (FF0.pin 1) -- ++(-1, 0) node[and port, number inputs=3,anchor=out] (and_d0) {};
    
 %   \draw (FF1.pin 4) -- ++(0.5, -0.5) -- ++(-1, 0) coordinate(Q1_bar_out) |- (and_d0.in 1);
 %   \draw (FF0.pin 4) -- ++(0.5, -0.5) -- ++(-2, 0) coordinate(Q0_bar_out) |- (and_d0.in 2);
 %   \draw (B) |- (and_d0.in 3);
    
 %   \draw (and_d0.out) -- (FF0.pin 1);
    
    % Output S (Same as D0 logic)
    \draw (and_d0.out) to[short,*-] ++(0, -3) coordinate(s_tap) -- ++(5, 0) node[right] {$S$};
    
    % Logic D1 = Q1' Q0 + Q1 Q0' B
    % Terms
    \draw (FF1.pin 1) -- ++(-1, 0) node[or port,anchor=out] (or_d1){};
    \draw (or_d1.in 1) -| ++ (-0.5,1) node[and port,anchor=out] (term1_d1) {}; % Q1' Q0
    \draw (or_d1.in 2) -| ++ (-0.5,-1) node[and port, number inputs=3,anchor=out] (term2_d1) {}; % Q1 Q0' B

    
    % Connections D1
    % Q1' Q0
 %   \draw (Q1_bar_out) |- (term1_d1.in 1);
 %   \draw (FF0.pin 6) -- ++(0.5, 0) coordinate(Q0_out) |- (term1_d1.in 2);
    
    % Q1 Q0' B
 %   \draw (FF1.pin 6) -- ++(0.5, 0) coordinate(Q1_out) |- (term2_d1.in 1);
%  \draw (Q0_bar_out) |- (term2_d1.in 2);
 %   \draw (B) |- (term2_d1.in 3);
    
    % OR for D1
 %   \draw (term1_d1.out) -- (or_d1.in 1);
 %   \draw (term2_d1.out) -- (or_d1.in 2);
 %   \draw (or_d1.out) -- (FF1.pin 1);

 \draw (FF1.pin 6) |- ++(-8.5,2.5) coordinate(Q1_out) |- (term2_d1.in 1);
 \draw (FF1.pin 4) -- ++(0.5,0)|- ++(-9.5,4.5) coordinate(Q1_bar_out) |- (and_d0.in 1);
 \draw (FF0.pin 6) |- ++(-9.5,0.8) |- (term1_d1.in 2);
 \draw (FF0.pin 4) -- ++(0.5,0)|- ++(-9,2.8) coordinate(Q0_bar_out) |- (term2_d1.in 2);

 \draw (term1_d1.in 1) to[short,-*] (term1_d1.in 1 -| Q1_bar_out);
 \draw (and_d0.in 2) -| (Q0_bar_out);
 \node[circ] at (Q0_bar_out) {};

 \draw (term2_d1.in 3) -- ++(-2.5,0) node[left] (B) {$B$};
 \node[circ] at ($(B) + (0.7,0)$) (B_tap) {};
 \draw (B_tap) |- (and_d0.in 3);

 \draw (FF1.pin 3) -- ++(-0.5,0) |- ++(-7.5,-5.7) node[left] {$CLK$};
 \draw (FF0.pin 3) to[short,-*] ++(-0.5,0);

 \draw (FF1.bdown) -- (FF1.down) -| ++(-1.2,-5.5) coordinate (Reset_tap) -- ++(-8,0) node[left] {Reset};
 \draw (FF0.bdown) -- (FF0.down) to[short,-*] (FF0.down -| Reset_tap);





\end{circuitikz}
\end{document}
