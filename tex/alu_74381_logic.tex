\documentclass[border=10pt]{standalone}
\usepackage[utf8]{inputenc}
\usepackage{circuitikz}
\usepackage{tikz}
\usetikzlibrary{shapes.geometric, positioning, calc}

\ctikzset{
logic ports=ieee,
logic ports/scale=1,
}

\begin{document}

\begin{circuitikz}
    % Global styles
    \tikzset{
        adder/.style={draw, rectangle, minimum size=1.2cm, fill=white},
        control/.style={font=\small, align=center}
    }
    
    % --- Frame ---
    \draw[thick] (-2, 0) rectangle (14, 11);

    % --- Inputs ---
    \node (a) at (-3, 9) {a};
    \node (b) at (-3, 3) {b};

    % --- Control Signal Labels (Top) ---
    \node[control] (ainv) at (1, 11.5) {AInvert};
    \node[control] (binv) at (4, 11.5) {BInvert};
    \node[control] (cin) at (8, 11.5) {CarryIn};
    \node[control] (op) at (12, 11.5) {Operation};

    % --- Input Multiplexers ---
    % MUX A
    \node[muxdemux, muxdemux def={Lh=4, NL=2, Rh=3, NB=0, w=1.5, NT=1, square pins=1}] (muxA) at (1, 9) {};
    \node [right][xshift=1em] at (muxA.lpin 1) {0};
    \node [right][xshift=1em] at (muxA.lpin 2) {1};
    
    % Connections for MUX A
    \draw (a) -- (-1, 9) coordinate (splitA);
    \draw (splitA) -- (splitA |- muxA.lpin 1) -- (muxA.lpin 1); % Direct input (0)
   % \draw (splitA) -- (-1, 8.4) node[not port, scale=0.6,anchor=in] (notA) {};
    \draw (muxA.lpin 2) -- ++(-0.2,0)node[not port, scale=0.6,anchor=out] (notA) {};
    \draw (notA.in) -| (splitA); % Inverted input (1)
    \draw (ainv) -- (muxA.tpin 1); % Control line

    % MUX B
    \node[muxdemux, muxdemux def={Lh=4, NL=2, Rh=3, NB=0, w=1.5, NT=1, square pins=1}] (muxB) at (1, 3) {};
    \node [right][xshift=1em] at (muxB.lpin 1) {0};
    \node [right][xshift=1em] at (muxB.lpin 2) {1};

    % Connections for MUX B
    \draw (b) -- (-1, 3) coordinate (splitB);
    \draw (splitB) -- (splitB |- muxB.lpin 1) -- (muxB.lpin 1); 
    %\draw (splitB) -- (-1, 2.2) node[not port, rotate=-90, scale=0.6] (notB) {};
    \draw (muxB.lpin 2) -- ++(-0.2,0)node[not port, scale=0.6,anchor=out] (notB) {};
    \draw (notB.in) -| (splitB); 
    \draw (notB.out) |- (muxB.lpin 2);
    
    % Control line for BInvert (goes down, bypasses Mux A)
    \draw (binv) |- (muxB.tpin 1);

    \node[circ] at (splitA) {};
    \node[circ] at (splitB) {};


    % --- Logic Gates & Adder ---
    % Vertical distribution
    \node[and port] (and) at (6, 8) {};
    \node[or port] (or) at (6, 6) {};
    \node[adder] (add) at (6, 4) {\Large +};
    \node[xor port] (xor) at (6, 2) {};

    % --- Internal Wiring (Buses) ---
    % Vertical rails for Mux outputs
    \draw (muxA.rpin 1) -- (2.5, 9) coordinate(railA) -- (2.5, 1.5); % Rail A
    \draw (muxB.rpin 1) -- (3.0, 3) node[circ] (railB) {} -- (3.0, 8.5); % Rail B

    % Connect Rail A to gates (Top inputs)
    \draw (railB) |- (xor.in 2);
    \draw (and.in 1) to[short,-*] (and.in 1 -| railA);
    \draw (or.in 1) to[short,-*] (or.in 1 -| railA);
    \path (add.west) -- ++(0,+0.2) coordinate (addIn1);
    \draw (addIn1) to[short,-*] (addIn1 -| railA);
    \draw (xor.in 1) to[short,-*] (xor.in 1 -| railA);
    
    % Connect Rail B to gates (Bottom inputs)
   % \draw (railA) |- (add.in 1);
    \draw (railB) |- (xor.in 2);
    \draw (and.in 2) to[short,-*] (and.in 2 -| railB);
    \draw (or.in 2) to[short,-*] (or.in 2 -| railB);

    \path (add.west) -- ++(0,-0.2) coordinate (addIn2);
    \draw (addIn2) to[short,-*] (addIn2 -| railB);
 %   \draw (xor.in 2) to[short,-*] (xor.in 2 -| railB);


    % --- Adder Specific Wiring ---
    % Carry In
    \draw (cin) -- (8, 4.8) -- (6, 4.8) -- (add.north);
    % Carry Out
    \draw (add.south) -- (6, 3) -- (8, 3) -- (8, -0.5) node[control, below] {CarryOut};
    \node[control, below] (Pi) at (9.5, -0.5) {Pi};
    \node[control, below] (Gi) at (10.5, -0.5) {Gi};
    %\draw (Pi) to[short,-*] (Pi |- xor.out);
    

    % --- Output Multiplexer ---
    \node[muxdemux, muxdemux def={Lh=14.5, NL=4, Rh=12.5, NB=0, w=2.0, NT=1, square pins=1}] (muxOut) at (12, 5) {};
    
    % Labels inside Big Mux
    \node [right, font=\small][xshift=1em] at (muxOut.lpin 1) {0};
    \node [right, font=\small][xshift=1em] at (muxOut.lpin 2) {1};
    \node [right, font=\small][xshift=1em] at (muxOut.lpin 3) {2};
    \node [right, font=\small][xshift=1em] at (muxOut.lpin 4) {3};
    \node[rotate=90, font=\small] at (muxOut.center) {Mux};

    % Operation Control
    \draw (op) -- (muxOut.tpin 1);

    % Connect Gate Outputs to Mux
    \draw (and.out) -- (muxOut.lpin 1); % Connect to level 0 area
    \draw (or.out) -- (muxOut.lpin 2);   % Connect to level 1 area
    \draw (add.east) -- (muxOut.lpin 3); % Connect to level 2 area
    \draw (xor.out) -- (muxOut.lpin 4); % Connect to level 3 area

    % --- Result Output ---
    \draw[->] (muxOut.rpin 1) -- (14.5, 5) node[right] {Result};

    % Connection Dots (Optional for clarity)
    
    \draw (Gi) to[short,-*] (Gi |- muxOut.lpin 1);
    \draw (Pi) to[short,-*] (Pi |- muxOut.lpin 4);


\end{circuitikz}

\end{document}