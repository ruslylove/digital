\documentclass[border=10pt]{standalone}
\usepackage{circuitikz}

\begin{document}
\begin{circuitikz}[american, scale=1.0]
    \ctikzset{logic ports=ieee} 
    % Flip-Flops
    \tikzset{d-ff/.style={flipflop, flipflop def={t1=D, t6=Q, t4=\ctikztextnot{Q}, c3=1}}}
    
    % Nodes
    \node (B) at (-2, 6) {$B$};
    \node (clk) at (6, -1) {CLK};
    
    % FFs
    \node[d-ff] (FF0) at (6, 1) {$Q_0$};
    \node[d-ff] (FF1) at (6, 5) {$Q_1$};
    
    % Clock
    \draw (clk) -| (FF0.pin 3);
    \draw (clk) -| (FF1.pin 3);
    
    % Logic D0 = S = Q1' Q0' B
    \node[and port, number inputs=3] (and_d0) at (2, 1) {};
    
    \draw (FF1.pin 4) -- ++(0.5, -0.5) -- ++(-1, 0) coordinate(Q1_bar_out) |- (and_d0.in 1);
    \draw (FF0.pin 4) -- ++(0.5, -0.5) -- ++(-2, 0) coordinate(Q0_bar_out) |- (and_d0.in 2);
    \draw (B) |- (and_d0.in 3);
    
    \draw (and_d0.out) -- (FF0.pin 1);
    
    % Output S (Same as D0 logic)
    \draw (and_d0.out) -- ++(1, 0) coordinate(s_tap) -- ++(0, -1.5) node[right] {$S$};
    
    % Logic D1 = Q1' Q0 + Q1 Q0' B
    % Terms
    \node[and port] (term1_d1) at (0, 7) {}; % Q1' Q0
    \node[and port, number inputs=3] (term2_d1) at (0, 5) {}; % Q1 Q0' B
    \node[or port] (or_d1) at (3, 6) {};
    
    % Connections D1
    % Q1' Q0
    \draw (Q1_bar_out) |- (term1_d1.in 1);
    \draw (FF0.pin 6) -- ++(0.5, 0) coordinate(Q0_out) |- (term1_d1.in 2);
    
    % Q1 Q0' B
    \draw (FF1.pin 6) -- ++(0.5, 0) coordinate(Q1_out) |- (term2_d1.in 1);
    \draw (Q0_bar_out) |- (term2_d1.in 2);
    \draw (B) |- (term2_d1.in 3);
    
    % OR for D1
    \draw (term1_d1.out) -- (or_d1.in 1);
    \draw (term2_d1.out) -- (or_d1.in 2);
    \draw (or_d1.out) -- (FF1.pin 1);

\end{circuitikz}
\end{document}
