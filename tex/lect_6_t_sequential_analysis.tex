\documentclass[border=10pt]{standalone}
\usepackage{circuitikz}
\usetikzlibrary{calc}

\begin{document}
\begin{circuitikz}[american, scale=1.0]
    \ctikzset{logic ports=ieee}
    
    % Inputs - Leftmost
    \node (clk_in) at (0.2, -5) {CLK};
    
    % Flip-Flops - Far Right
    % Custom T Flip Flop shape isn't standard in basic circuitikz sometimes, 
    % but we can use generic flipflop or JK shorted. 
    % Actually circuitikz has nice flipflop wrappers now.
    % Let's use the standard flipflop node with t labels if needed, or just shape=flipflop T using newer libs.
    % To match previous style, we use "flipflop T" if available, or generic. Assuming fairly modern circuitikz based on earlier files.
    % Previous file used "flipflop D" and "flipflop JK".
    \node[flipflop T] (TFF_A) at (10, 2) {};

    \node[flipflop T] (TFF_B) at (10, -3) {};


  
    
    % Logic for TA = Bx
    % AND gate (B, x)
    \path (TFF_A.pin 1) -- ++(-3, 0) coordinate (ta_pos);
    \node[and port, anchor=out] (AND_TA) at (ta_pos) {$Bx$};
    \draw (AND_TA.out) -- (TFF_A.pin 1);

  % Input x
    \draw (AND_TA.in 2) to[short, -*] ++(-2, 0) coordinate (x_bus_top) -- ++(-1,0) node[left] {$x$};
   % \node (x_in_node) at (-2, 0) {$x$};
   % \draw (x_in_node) -- ++(2,0) coordinate (x_bus_top);
    \node[circ] at (x_bus_top) {};
    
    % Logic for TB = x
    % Direct connection from x
    \draw (TFF_B.pin 1) -- ++(-1, 0) coordinate (TB_in);
    
    % Wiring Inputs
    % TB = x
    % x_bus is at x=0
    \draw (x_bus_top) |- (TB_in) -- (TFF_B.pin 1);
    
    % Connect x to AND_TA (input 2)
    \draw (x_bus_top |- AND_TA.in 2) to[short, *-] (AND_TA.in 2);
    
    % Feedback Connections
    % B (TFF_B.pin 6)
    \draw (TFF_B.pin 6) -- ++(1, 0) coordinate (B_out);
    \node[above] at (B_out) {$B$};
    \path (TFF_B.pin 6) -- ++(0.3, 0) coordinate (B_common);
    
    % B to AND_TA (input 1)
    % Loop UP 
    \draw (B_common) to[short, *-] ++(0, 6.2) coordinate (B_top) -| (AND_TA.in 1);
    
    % Output y = AB
    % AND gate connected to A and B
    % Place it to the right of FFs
    \path (TFF_A.pin 6) -- ++(2, 0) coordinate (y_pos);
    \node[and port, anchor=in 1] (AND_y) at (y_pos) {$AB$};
    
    % A (TFF_A.pin 6)
    \draw (TFF_A.pin 6) -- ++(1, 0) coordinate (A_out);
    \node[above] at (A_out) {$A$};
    
    % Route A to AND_y (input 1)
    \draw (A_out) -| (AND_y.in 1);
    
    % Route B to AND_y (input 2)
    \draw (B_out) -| (AND_y.in 2);
    
    % Draw Output y
    \draw (AND_y.out) -- ++(1, 0) node[right] {$y$};

    % Clock Distribution
    \draw (TFF_B.pin 3) -- ++(-0.5, 0) node[circ] (clk_node) {};
    \draw (clk_in) -| (clk_node);
    \draw (clk_node) |- (TFF_A.pin 3);

\end{circuitikz}
\end{document}
