\documentclass{standalone} % 'standalone' is good for generating just the circuit image
\usepackage[utf8]{inputenc}
\usepackage{circuitikz}

\begin{document}
\begin{circuitikz}[american] % Using 'american' style for components

    % 1. Define the component labels
    \def\SwLabel{S}
    \def\GateLabel{G}

    % 2. Draw the main circuit line (Source and Drain path)
    \draw (0, 0) to[short, *-] (0, 3) node[above] {Drain (D)};

    % 3. Draw the N-Channel MOSFET (Ideal Switch)
    \draw (0, 3) node[anchor=south] {}
          to[R, l_={R\_load}] (0, 4.5); % Add a load for context
    \draw (0, 4.5) to[short, -*] (0, 6); % Connect to power

    % Draw the MOSFET (nmos or a generic controlled switch)
    \draw (0, 3) -- (2, 3)
          node[nmos, anchor=D, bulk=no, bodydiode=no] (Mosfet) {} % nmos symbol
          (2, 3) -- (2, 0);

    % 4. Add the gate control signal and label
    \draw (Mosfet.G) -- ++(1.5, 0) node[right] {Control Signal};
    \draw (Mosfet.G) node[left] {Gate ($\mathbf{\mathbf{G}}$)};
    
    % 5. Add terminal labels (Source and Drain)
    \draw (Mosfet.D) node[above right] {Drain ($\mathbf{\mathbf{D}}$)};
    \draw (Mosfet.S) node[below right] {Source ($\mathbf{\mathbf{S}}$)};

    % 6. Indicate the Switch State
    \draw[->, dashed, color=blue] (1, 3.5) -- (1, 4.5) node[midway, right] {I};
    \node[draw, minimum width=2.5cm, minimum height=1cm, align=center, fill=blue!10] 
          at (3.5, 5.5) {Ideal Switch \\ (N-MOSFET Representation)};

    % Optional: Add a simple general controlled switch symbol if preferred
    % \draw (5, 0) to [cute open switch, name=sw, l=\SwLabel] (5, 3);
    % \draw (sw.wiper) -- ++(1, 0) node[right] {Gate $\mathbf{\mathbf{G}}$};
    % \draw (sw.out 1) node[left] {Terminal 1};
    % \draw (sw.out 2) node[left] {Terminal 2};

\end{circuitikz}
\end{document}