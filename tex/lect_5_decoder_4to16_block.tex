\documentclass[border=10pt]{standalone}
\usepackage[american]{circuitikz}
\usetikzlibrary{calc}

\begin{document}
\begin{circuitikz}
    \ctikzset{logic ports=ieee}
    % Styles
    \tikzset{block/.style={draw, rectangle, minimum width=3cm, minimum height=4cm, align=center}}

    % Decoders
    % Top Decoder (D0-D7)
    \node[block] (dec1) at (4, 4) {3-to-8\\Decoder};
    
    % Bottom Decoder (D8-D15)
    \node[block] (dec2) at (4, -2) {3-to-8\\Decoder};

    % Inputs (x, y, z) - Aligned with Dec 1 inputs A, B, C
    \node (x) at (0, 4.5) {$x$};
    \node (y) at (0, 4.0) {$y$};
    \node (z) at (0, 3.5) {$z$};

    % Route x, y, z to internal pins
    % Dec 1 inputs (Top) - Straight connections
    \node[right] at ($(dec1.west)+(0, 0.5)$) {A}; % y=4.5
    \node[right] at ($(dec1.west)+(0, 0)$) {B};   % y=4.0
    \node[right] at ($(dec1.west)+(0, -0.5)$) {C}; % y=3.5
    
    \draw (x) -- ($(dec1.west)+(0, 0.5)$);
    \draw (y) -- ($(dec1.west)+(0, 0)$);
    \draw (z) -- ($(dec1.west)+(0, -0.5)$);

    % Dec 2 inputs (Bottom) - Route down
    \node[right] at ($(dec2.west)+(0, 0.5)$) {A};
    \node[right] at ($(dec2.west)+(0, 0)$) {B};
    \node[right] at ($(dec2.west)+(0, -0.5)$) {C};

    \draw (1, 4.5) |- ($(dec2.west)+(0, 0.5)$);
    \draw (1.2, 4.0) |- ($(dec2.west)+(0, 0)$);
    \draw (1.4, 3.5) |- ($(dec2.west)+(0, -0.5)$);
    
    % Connection dots
    \node[circ] at (1, 4.5) {};
    \node[circ] at (1.2, 4.0) {};
    \node[circ] at (1.4, 3.5) {};


    % Input w (MSB) - Controls Enable
    \node (w) at (0, 5) {$w$};
    
    % Enable pins (Active Low usually)
    % Let's assume Active Low E, consistent with previous slide (E=0 to enable).
    % If w=0, D0-D7 active. So w->E1.
    % If w=1, D8-D15 active. So w->NOT->E2.
    
    \node[right] at ($(dec1.west)+(0, -1.5)$) {$E$}; % Pin label
    %\node[circ] at ($(dec1.west)+(0, -1.5)$) {}; % Inversion bubble visual if desired, closer to body? 
                                                  % Standard circuit rule: Active low usually implies bubble. 
                                                  % Text says "Enable input (E)". Previous text: E=1 disabled.
                                                  % So E=0 enabled.
    
    \node[right] at ($(dec2.west)+(0, -1.5)$) {$E$} ;

    % Connect w to Dec1 E
    %\draw (w) -| ($(dec1.west)+(-1, -1.5)$) -- ($(dec1.west)+(0, -1.5)$);
    \draw (w) -- ++(1.6,0) coordinate (w_e);
    \draw (w_e) |- ($(dec2.west)+(0, -1.5)$);
    \draw ($(dec1.west)+(0, -1.5)$) to[short, o-*] ++(-0.9, 0); 
    
    %\draw (dec2_e) -- (dec2_e -| w_e);

    % Connect w to Dec2 E via Inverter
    % Inverter
    % \node[not port,  scale=0.5] (inv) at (2, 2.4) {}; % Position between?
    % Let's verify routing. w is at y=6. Dec2 E is at y=-3.5.
    
    % Draw w down to inverter input
    % \draw (w) -- (0, 6) -- (2.5, 6) -- (inv.in);
    %\draw (inv.out) |- ($(dec2.west)+(0, -1.5)$);
    %\node[circ] at (2.5, 6) {}; % Junction at w line (actually w source is node w)

    
    % Outputs
    % Dec 1 (D0-D7)
    \foreach \i in {0,...,7} {
        \draw ($(dec1.east)+(0, 1.4 - \i*0.4)$) -- ++(1, 0) node[right] {$D_{\i}$};
    }

    % Dec 2 (D8-D15)
    \foreach \i [count=\j from 8] in {0,...,7} {
        \draw ($(dec2.east)+(0, 1.4 - \i*0.4)$) -- ++(1, 0) node[right] {$D_{\j}$};
    }
    
\end{circuitikz}
\end{document}
