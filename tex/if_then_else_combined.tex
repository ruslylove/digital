\documentclass[border=10pt]{standalone}
\usepackage{tikz}
\usepackage{circuitikz}
\ctikzset{logic ports=ieee}
\usetikzlibrary{positioning, calc, decorations.markings, shapes.geometric}

% Define the register style similar to counter_datapath.tex
\tikzset{register_style/.style={muxdemux, muxdemux def={Lh=3, Rh=3, NL=2, NB=1, NT=1, w=5.0, NR=0},
                            muxdemux label={L1=Load, L2=Clock, L3=Clear, T1=$D$, B1=$Q$}}}

\begin{document}

\begin{circuitikz}[
    font=\sffamily,
    arrow/.style={-Latex, line width=1pt},
    label_text/.style={font=\footnotesize},
    bus/.style={-Latex, line width=1.5pt,
        postaction={decorate},
        decoration={markings, mark=at position 0.5 with {
            \draw[thick,-] (-2pt,-3pt) -- (2pt,3pt);
            \node[above left=0.5pt] {\footnotesize $4$}; 
        }}},
    control_signal/.style={-Latex, line width=0.5pt},
]

    % --- Control Unit ---
    % Inputs (Left): Clock, Reset
    % Inputs (Right): (A=5)
    % --- Control Unit ---
    % Inputs (Left): Clock, Reset
    % Inputs (Right): (A=5)
    % Outputs (Right): ALoad, Muxsel, BLoad, Out (Reordered for routing)
    \node[muxdemux, muxdemux def={Lh=8, Rh=8, NL=2, NR=5, NB=0, NT=0, w=8},
          muxdemux label={L1=Clk, L2=Rst, 
                          R1=ALoad, R2=Muxsel, R3=BLoad, R4=Out, R5={$(A=5)$}}] (CU) at (-7, 1) {Control\\Unit};

    % CU Inputs
    \draw[control_signal] ($(CU.blpin 1)+(-1,0)$) node[left] {Clock} -- (CU.blpin 1);
    \draw[control_signal] ($(CU.blpin 2)+(-1,0)$) node[left] {Reset} -- (CU.blpin 2);

    % --- Register A ---
    \node[register_style, muxdemux label={L1=, L2=, L3=, T1=In}, 
    muxdemux label={T1=$D$, L1=$Load$, L2={$Clock$}}, anchor=blpin 1] (RegA) at (0,4.5) {Reg A};
    % Shift RegA slightly to align pin 1 with ALoad channel if needed, sticking to (0,4.5) to give space for Comp connection.
    % Actually default was (0,4). Let's keep components roughly same, adjust lines.
    
    % Input arrow to Reg A
    \node[above] at ($(RegA.btpin 1)+(0,1.5)$) (Input) {Input};
    \draw[bus] (Input) -- (RegA.btpin 1);

    % --- Comparator (A vs 5) ---
    \node[muxdemux, muxdemux def={Lh=2, Rh=2, NL=0, NB=0, NT=2, w=3.5, NR=1},
          muxdemux label={T1=$A$, T2=$B$, R1={$(A=5)$}}, anchor=btpin 1] (Comp) at ($(RegA.bpin 1)+(0,-3.4)$) {};
    
    % Connect Reg A to Comp A input (tpin 1 -> btpin 1)
    \draw[bus] (RegA.bbpin 1) -- ++(0,-1.5) -| (Comp.btpin 1);
    
    % Connect Constant 5 to Comp B input (tpin 2 -> btpin 2)
    \draw[bus] ($(Comp.btpin 2)+(0,1)$) node[above] {$'5'$} -- (Comp.btpin 2);
    
    % Status Signal: Comp (A=5) -> CU
    % Route: Comp (0,0) -> Down to y=-2.5 -> Left -> Up to CU.R5 (y~ -2)
    \draw[control_signal] (Comp.rpin 1) |- ++(-4, -1)  |- (CU.brpin 5);

    % --- Mux (Trapezoidal 2-to-1) ---
    \node[muxdemux, muxdemux def={Lh=6, Rh=3, NL=2, NB=1, NT=0, w=2, NR=1},
          muxdemux label={L1=$0$, L2=$1$, B1=$Sel$},
          rotate=-90] (Mux) at (6, 4) {Mux};
          
    % Mux Inputs
    \draw[bus] ($(Mux.blpin 1)+(0,1.5)$) node[left] {$'8'$} -- (Mux.blpin 1);
    \draw[bus] ($(Mux.blpin 2)+(0,1.5)$) node[left] {$'13'$} -- (Mux.blpin 2);

    % --- Register B ---
    \node[register_style] (RegB) at (6, 0) {Reg B};
    
    % Connect Mux Out to Reg B In
    \draw[bus] (Mux.brpin 1) -- (RegB.btpin 1);

    % --- Tri-state Buffer ---
    \node[buffer port, rotate=-90, yscale=-1, scale=0.5] (Tri) at (6, -3) {};
    
    % Connect Reg B Out to Tri-state In
    \draw[bus] (RegB.bbpin 1) -- (Tri.bin);
    
    % Tri-state Output
    \draw[bus] (Tri.bout) -- ++(0,-1) node[below] {$Output$};
    
    % --- Control Signal Connections (Optimized Routing) ---
    
    % ALoad: CU.R1 -> RegA.L1
    % Direct path
    \draw[control_signal] (CU.brpin 1) -- (CU.rpin 1) -- ++(3,0) |- (RegA.blpin 1);
    
    % Muxsel: CU.R2 -> Mux.B1
    % Route HIGH (y=6) to avoid RegA-Comp bus
    \draw[control_signal] (CU.brpin 2) -- ++(0.5,0) |- ++(0, 0.5) -- ++(8,0) |- (Mux.bpin 1) -- (Mux.bbpin 1);
    
    % BLoad: CU.R3 -> RegB.L1
    % Route LOW (y=-1.5) to avoid crossing data lines
    % Target RegB is at y=0. Lpin 1 is top.
    \draw[control_signal] (CU.brpin 3) -- ++(1.0,0) |- (0, -2.5) -| ($(RegB.blpin 1)+(-1,0)$) -- (RegB.blpin 1);
    
    % Out: CU.R4 -> Tri.up
    % Route BOTTOM (y=-4)
    \draw[control_signal] (CU.brpin 4) -- ++(0.5,0) |- (Tri.up);
    
    % Clock Line (Common)
    \coordinate (clk_source) at ($(CU.blpin 1)+(-0.5,0)$);
    \fill [black] (clk_source) circle (2pt);
    
    % Route Clock via very bottom or top?
    % Let's go very bottom (y=-5)
    \draw[control_signal] (clk_source) -- ++(0,-6) coordinate (clk_bottom) -- ++(14,0) |- (RegB.blpin 2);
    % Connect RegB clock
    \draw[control_signal] (RegA.lpin 2 |- clk_bottom) to[short,*-] (RegA.lpin 2) -- (RegA.blpin 2);


\end{circuitikz}
\end{document}
