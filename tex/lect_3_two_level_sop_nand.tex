\documentclass[border=10pt]{standalone}
\usepackage{circuitikz}

\ctikzset{
logic ports=ieee,
logic ports/scale=0.7,
}

\begin{document}
\begin{circuitikz}[]

    % ============================
    % 1. FIRST LEVEL (Left Side)
    % ============================
    
    % Top Gate: NAND for Inputs A and B
    % Replaces the term (AB)
    \draw (0, 4) node[nand port] (NAND_AB) {};
    \draw (NAND_AB.in 1) -- ++(-1, 0) node[left] {$A$};
    \draw (NAND_AB.in 2) -- ++(-1, 0) node[left] {$B$};

    % Middle Gate: Inverter for Input E
    % Ideally, E should enter the final NAND as E_bar to become E in the output.
    % We use a NAND wired as a NOT gate.
    \draw (0, 2) node[nand port] (NAND_E) {};
    % Connect input E to both pins of this NAND
    \draw (NAND_E.in 1) -- ++(-0.5, 0) coordinate(splitE) -- ++(-0.5, 0) node[left] {$E$};
    \draw (splitE) node[circ]{} |- (NAND_E.in 2);

    % Bottom Gate: NAND for Inputs C and D
    % Replaces the term (CD)
    \draw (0, 0) node[nand port] (NAND_CD) {};
    \draw (NAND_CD.in 1) -- ++(-1, 0) node[left] {$C$};
    \draw (NAND_CD.in 2) -- ++(-1, 0) node[left] {$D$};


    % ============================
    % 2. SECOND LEVEL (Right Side)
    % ============================

    % Final Gate: 3-Input NAND
    % Replaces the final OR gate
    \draw (3, 2) node[nand port, number inputs=3] (FINAL) {};


    % ============================
    % 3. CONNECTIONS
    % ============================
    
    % Connect Top NAND to Final Input 1
    \draw (NAND_AB.out) -- ++(0.5,0) |- (FINAL.in 1);

    % Connect Middle NAND (Inverter) to Final Input 2
    \draw (NAND_E.out) -- (FINAL.in 2);

    % Connect Bottom NAND to Final Input 3
    \draw (NAND_CD.out) -- ++(0.5,0) |- (FINAL.in 3);


    % ============================
    % 4. OUTPUT
    % ============================
    \draw (FINAL.out) -- ++(1, 0) node[right] {$F = AB + CD + E$};

\end{circuitikz}
\end{document}