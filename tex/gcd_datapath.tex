\documentclass[border=10pt]{standalone}
\usepackage{tikz}
\usepackage{circuitikz}
\ctikzset{logic ports=ieee}
\usetikzlibrary{positioning, calc, decorations.markings, shapes.geometric}

% Define the register style
\tikzset{register_style/.style={muxdemux, muxdemux def={Lh=3, Rh=3, NL=2, NB=1, NT=1, w=5.0, NR=0},
                            muxdemux label={L1=Load, L2=Clock, L3=Clear, T1=$D$, B1=$Q$}}}

\begin{document}

\begin{circuitikz}[
    font=\sffamily,
    arrow/.style={-Latex, line width=1pt},
    label_text/.style={font=\small},
    bus/.style={-Latex, line width=1.5pt,
        postaction={decorate},
        decoration={markings, mark=at position 0.5 with {
            \draw[thick,-] (-2pt,-3pt) -- (2pt,3pt);
            %\node[above left=0.5pt] {\footnotesize $n$}; 
        }}},
    control_signal/.style={-Latex, line width=0.5pt, color=blue},
]
    % --- Layout Variables ---
    \def\regY{4}      % Y position of Registers
    \def\muxInY{7}    % Y position of Input Muxes
    \def\muxOpY{0}    % Y position of Operand Muxes
    \def\subY{-3}     % Y position of Subtractor
    \def\colX{0}      % X column for X path
    \def\colY{8}      % X column for Y path
    
    % --- Control Signals List (Left Side) ---
    \node[anchor=north west, align=left] at (-8, 8) {
        \textbf{Control Signals}:\\
        \textcolor{blue}{In\_X}: $1=$ Input, $0=$ Sub\\
        \textcolor{blue}{In\_Y}: $1=$ Input, $0=$ Sub\\
        \textcolor{blue}{XLoad}: Load X\\
        \textcolor{blue}{YLoad}: Load Y\\
        \textcolor{blue}{XY}: $0=X-Y$, $1=Y-X$\\
        \textcolor{blue}{Out}: Output Enable
    };

    % --- Input Muxes (Select: Input vs Feedback) --
    % MuxInX
    \node[muxdemux, muxdemux def={Lh=4, Rh=3, NL=2, NB=1, NT=0, w=2, NR=1},
          muxdemux label={L1=$Sub$, L2=$In$, B1=$Sel$},
          rotate=-90] (MuxInX) at (\colX, \muxInY) {Mux X};
    % MuxInY
    \node[muxdemux, muxdemux def={Lh=4, Rh=3, NL=2, NB=1, NT=0, w=2, NR=1},
          muxdemux label={L1=$Sub$, L2=$In$, B1=$Sel$},
          rotate=-90] (MuxInY) at (\colY, \muxInY) {Mux Y};

    % --- Registers ---
    \node[register_style, anchor=btpin 1] (RegX) at (\colX, \regY) {Reg X};
    \node[register_style, anchor=btpin 1] (RegY) at (\colY, \regY) {Reg Y};
    
    % Connect MuxIn -> Reg
    \draw[bus] (MuxInX.brpin 1) -- (RegX.btpin 1);
    \draw[bus] (MuxInY.brpin 1) -- (RegY.btpin 1);
    
    % --- System Inputs ---
    \draw[bus] ($(MuxInX.blpin 2)+(0,1.5)$) node[left] {Input X} -- (MuxInX.blpin 2);
    \draw[bus] ($(MuxInY.blpin 2)+(0,1.5)$) node[left] {Input Y} -- (MuxInY.blpin 2);

    % --- Output Signal Routing ---
    \coordinate (Xout) at ($(RegX.bbpin 1)+(0,-1)$);
    \coordinate (Yout) at ($(RegY.bbpin 1)+(0,-1)$);
    
    \draw[bus] (RegX.bbpin 1) -- (Xout);
    \draw[bus] (RegY.bbpin 1) -- (Yout);

    % --- Comparator ---
    \node[muxdemux, muxdemux def={Lh=2.5, Rh=2.5, NL=0, NB=0, NT=2, w=4, NR=2},
          muxdemux label={T1=$A$, T2=$B$, R1={$(A=B)$}, R2={$(A>B)$}}] (Comp) at (4, 2) {Comp};
          
    \draw[bus] (Xout) -| ($(Comp.btpin 1)+(0,0.5)$) -- (Comp.btpin 1);
    \draw[bus] (Yout) -| ($(Comp.btpin 2)+(0,0.5)$) -- (Comp.btpin 2);

    % --- Operand Muxes (For swapping) ---
    % MuxOpA (Feeds Subtractor A)
    % 0: X, 1: Y
    \node[muxdemux, muxdemux def={Lh=4, Rh=3, NL=2, NB=1, NT=0, w=2, NR=1},
          muxdemux label={L1=$X$, L2=$Y$, B1=$Sel$},
          rotate=-90] (MuxOpA) at (2, \muxOpY) {Op A};
          
    % MuxOpB (Feeds Subtractor B)
    % 0: Y, 1: X
    \node[muxdemux, muxdemux def={Lh=4, Rh=3, NL=2, NB=1, NT=0, w=2, NR=1},
          muxdemux label={L1=$Y$, L2=$X$, B1=$Sel$},
          rotate=-90] (MuxOpB) at (6, \muxOpY) {Op B};
          
    % Routing X to MuxOpA(0) and MuxOpB(1)
    \draw[bus] (Xout) |- ($(MuxOpA.blpin 1)+(0,1)$) -- (MuxOpA.blpin 1);
    \draw[bus] (Xout) -- ++(0,-1.5) -| ($(MuxOpB.blpin 2)+(0,1)$) -- (MuxOpB.blpin 2);
    
    % Routing Y to MuxOpA(1) and MuxOpB(0)
    \draw[bus] (Yout) -- ++(0,-1.5) -| ($(MuxOpA.blpin 2)+(0,1)$) -- (MuxOpA.blpin 2);
    \draw[bus] (Yout) |- ($(MuxOpB.blpin 1)+(0,1)$) -- (MuxOpB.blpin 1);

    % --- Subtractor ---
    \node[muxdemux, muxdemux def={Lh=3, Rh=3, NL=0, NB=1, NT=2, w=4, NR=0},
          muxdemux label={T1=$A$, T2=$B$, B1=$Res$}] (Sub) at (4, \subY) {Subtractor};
          
    \draw[bus] (MuxOpA.brpin 1) -- (Sub.btpin 1);
    \draw[bus] (MuxOpB.brpin 1) -- (Sub.btpin 2);

    % --- Feedback Loop ---
    % Sub Output -> MuxInX(0) and MuxInY(0)
    \draw[bus] (Sub.bbpin 1) -- ++(0,-1) coordinate (SubOut);
    
    % Feedback to X
    \draw[bus] (SubOut) -| (-3, 0) |- ($(MuxInX.blpin 1)+(0,1.5)$) -- (MuxInX.blpin 1);
    
    % Feedback to Y
    \draw[bus] (SubOut) -| (11, 0) |- ($(MuxInY.blpin 1)+(0,1.5)$) -- (MuxInY.blpin 1);

    % --- Output Tri-state ---
    \node[buffer port, rotate=-90, scale=0.8] (Tri) at (-2, \regY) {};
    % Connect X to Tri in
    \draw[bus] (RegX.lpin 1) -- ++(-1,0) -- (Tri.in);
    \draw[bus] (Tri.out) -- ++(-1,0) node[left] {Output System};
    
    % --- Control Signals Drawing ---
    
    % In_X
    \draw[control_signal] ($(MuxInX.bbpin 1)+(0, 0.5)$) node[right] {In\_X} -- (MuxInX.bbpin 1);
    % In_Y
    \draw[control_signal] ($(MuxInY.bbpin 1)+(0, 0.5)$) node[left] {In\_Y} -- (MuxInY.bbpin 1);
    
    % XLoad
    \draw[control_signal] ($(RegX.blpin 1)+(-1,0)$) node[left] {XLoad} -- (RegX.blpin 1);
    % YLoad
    \draw[control_signal] ($(RegY.blpin 1)+(-1,0)$) node[left] {YLoad} -- (RegY.blpin 1);
    
    % XY Signal (Shared)
    \coordinate (XYNode) at (4, \muxOpY);
    \draw[control_signal] (XYNode) node[above=1.2] {XY} -- (MuxOpA.bbpin 1);
    \draw[control_signal] (XYNode) -- (MuxOpB.bbpin 1);
    
    % Out Signal
    \draw[control_signal] (Tri.up) -- ++(0, 0.5) node[above] {Out};
    
    % Clock
    \draw[control_signal] (RegX.blpin 2) -- ++(-0.5,0) node[left] {Clk};
    \draw[control_signal] (RegY.blpin 2) -- ++(-0.5,0) node[left] {Clk};

\end{circuitikz}
\end{document}
