\documentclass[border=10pt]{standalone}
\usepackage{karnaugh-map}

\begin{document}

% [4][4] specifies a 4-variable map (4x4 grid)
% [$CD$] labels the columns (Top)
% [$AB$] labels the rows (Left)
\begin{karnaugh-map}[4][4][1][$yz$][$wx$]

    % 1. Place the '1's (Minterms)
    % You list the decimal position of the cells that should be 1.
    % Note: The package automatically puts them in the correct Gray code spot.
    \minterms{0, 1, 2, 4, 5, 6, 12, 8, 9, 13, 14}

    % 2. Place the '0's (Maxterms) - Optional, often left blank
    \maxterms{ 3, 7, 10, 11, 15}

    % 3. Draw the loops (Implicants)
    % Syntax: \implicant{cell_start}{cell_end}
    
    % A square group of 4 (covering cells 0, 1, 4, 5)
    \implicant{0}{9} 

    % A grouping of the corners (wrapping around)
    % Using corner cells 0 and 10 usually creates a corner group
    % But here let's group the vertical wrap of 8 and 12
    \implicantedge{0}{4}{2}{6}

    \implicantedge{4}{12}{6}{14}



\end{karnaugh-map}

\end{document}