\documentclass[border=10pt]{standalone}
\usepackage{circuitikz}
\usetikzlibrary{calc}

\begin{document}
\begin{circuitikz}[american, scale=1.0]
    \ctikzset{logic ports=ieee}
    
    % Inputs
    \node (clk_in) at (-0.5, -2) {CLK};
    
    % Flip-Flops A0, A1, A2 from Left to Right
    
    % Flip-Flop A0 (LSB)
    \node[flipflop T] (TF_0) at (2, 0) {};
    \node[above] at (TF_0.n) {$A_0$};
    
    % Flip-Flop A1
    \node[flipflop T] (TF_1) at (7, 0) {};
    \node[above] at (TF_1.n) {$A_1$};
    
    % Flip-Flop A2 (MSB)
    \node[flipflop T] (TF_2) at (12, 0) {};
    \node[above] at (TF_2.n) {$A_2$};

    % T Inputs
    
    % T_A0 = 1
    \draw (TF_0.pin 1) -- ++(-1, 0) coordinate (logic1);
    \node[left] at (logic1) {$'1'$}; 

    % T_A1 = A0
    % Logic Level above: y = 2.5
   % \draw (TF_0.pin 6) -- ++(0.5, 0) coordinate (q0_out);
    
    % Route Q0 Up and Over to T1
    %\draw (q0_out) to[short, -*] ++(0, 2.5) coordinate (q0_bus) -- (q0_bus -| TF_1.pin 1) -- (TF_1.pin 1);
    %\node[above] at (q0_out |- q0_bus) {$A_0$};

    % T_A2 = A1A0
    % AND gate for T_A2 placed above TF_2 area, or between TF_1 and TF_2
    % Let's place it aligned with T input of TF_2 but higher? 
    % Or simply place it comfortably above.
    \draw (TF_1.pin 6) -- ++(0.5, 0) node[and port, anchor=in 2, scale=1] (AND_TA2) {$A_1A_0$};
    
    % Connect AND output to T_A2
    \draw (AND_TA2.out) -| (TF_2.pin 1);
    \draw (TF_0.pin 6) -- (TF_1.pin 1);

    \path (TF_1.pin 1) -- ++(-1.5,0) node[circ] (A0_out){};
    \draw (AND_TA2.in 1) -- ++(0,0.75) -| (A0_out);
    % Inputs to AND_TA2
    % Input 2: A0 (from q0_bus)
    % q0_bus is at y=2.5. AND_TA2 is at y=2.5.
    %\draw (q0_bus) -- (AND_TA2.in 2);
    
    % Input 1: A1 (from TF_1.Q)
    %\draw (TF_1.pin 6) -- ++(0.5, 0) coordinate (q1_out);
    %\draw (q1_out) |- (AND_TA2.in 1);
    %\node[right] at (q1_out |- AND_TA2.in 1) {$A_1$};

    % Outputs Indication
   % \draw (TF_0.pin 6) ++(0.5, 0) to[short, *-] ++(0, 1.5) node[above] {};
   % \draw (TF_1.pin 6) ++(0.5, 0) to[short, *-] ++(0, 1.5) node[above] {};
   % \draw (TF_2.pin 6) -- ++(1, 0) to[short, -] ++(0, 1.5) node[above, align=center] {$A_2$\\(MSB)};
   % \node[above, align=center] at (TF_0.pin 6 |- q0_bus) [yshift=1cm] {$A_0$\\(LSB)};
   % \node[above, align=center] at (TF_1.pin 6 |- q0_bus) [yshift=1cm] {$A_1$};
    
    % Clock Distribution
    \draw (clk_in) -- ++(13, 0) coordinate (clk_rail);
    
    \draw (TF_0.pin 3) -- (TF_0.pin 3 |- clk_rail) node[circ] {};
    \draw (TF_1.pin 3) -- (TF_1.pin 3 |- clk_rail) node[circ] {};
    \draw (TF_2.pin 3) -- (TF_2.pin 3 |- clk_rail) node[circ] {};
    
\end{circuitikz}
\end{document}
