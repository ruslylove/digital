\documentclass[border=10pt]{standalone}
\usepackage{circuitikz}
\usetikzlibrary{arrows,positioning}

\begin{document}

\begin{tikzpicture}[
    % Style for logic gates
    logigate/.style={
        logic port,
        scale=0.8
    },
    % Style for small junction circles
    circ/.style={
        circle, 
        fill, 
        inner sep=1pt
    }
]

    % --- GATES ---
    % Place the gates using relative positioning for easy adjustments
    \node[not port, logigate] (not1) at (0, 2.5) {};
    \node[and port, logigate] (and1) at (2, 3) {};
    \node[and port, logigate] (and2) at (2, 1) {};
    \node[or port, logigate]  (or1)  at (4, 2) {};

    % --- INPUTS ---
    \node (s_in)  at (-2, 2.5) {$s$};
    \node (x1_in) at (-2, 3.5) {$x_1$};
    \node (x2_in) at (-2, 0.5) {$x_2$};

    % --- CONNECTIONS ---
    \draw (s_in) -- (not1.in);
    \draw (not1.out) -- (and1.in 1);
    \draw (x1_in) -- (and1.in 2);
    \draw (s_in) -| (and2.in 1); % Wire s to the second AND gate
    \node at (s_in -| and2.in 1) [circ] {}; % Junction point for s
    \draw (x2_in) -- (and2.in 2);
    \draw (and1.out) -- (or1.in 1);
    \draw (and2.out) -- (or1.in 2);

    % --- OUTPUT ---
    \draw (or1.out) -- ++(1,0) node[right] {$f = s'x_1 + sx_2$};

\end{tikzpicture}

\end{document}