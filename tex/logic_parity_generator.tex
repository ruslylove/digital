\documentclass[border=10pt]{standalone}
% We continue to use the [ieee] option for standard curved logic gates.
\usepackage{circuitikz}

\ctikzset{
logic ports=ieee,
logic ports/scale=0.7,
}

\begin{document}

\begin{circuitikz}[]

    % ============================
    % 1. Place the XOR Gates
    % ============================
    % Gate 1: Computes (x XOR y). Placed on the left.
    \draw (2, 2) node[xor port] (XOR1) {};
    
    % Gate 2: Computes (Result XOR z). Placed to the right and lower down.
    \draw (6, 1) node[xor port] (XOR2) {};


    % ============================
    % 2. Draw Inputs x and y
    % ============================
    % Connect directly to the first XOR gate.
    \draw (XOR1.in 1) -- ++(-1.5, 0) node[left] {$x$};
    \draw (XOR1.in 2) -- ++(-1.5, 0) node[left] {$y$};


    % ============================
    % 3. Draw Input z and Inter-stage Connections
    % ============================
    
    % Connection between gates:
    % Output of XOR1 goes to the top input of XOR2.
    % We use '-|' to draw a horizontal-then-vertical path.
    \draw (XOR1.out) -- ++(0.5,0) -| (XOR2.in 1);

    % Input z:
    % It runs below the first gate and connects to the bottom input of XOR2.
    % We start it aligned with the other inputs, but lower at y=0.
    \draw (0,0) node[left] {$z$} -- ++(3,0) |- (XOR2.in 2);


    % ============================
    % 4. Final Output
    % ============================
    \draw (XOR2.out) -- ++(1, 0) node[right] {$P = x \oplus y \oplus z$};

\end{circuitikz}

\end{document}