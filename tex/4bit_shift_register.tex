\documentclass[border=10pt]{standalone}
\usepackage{circuitikz}
\usetikzlibrary{calc}

\begin{document}
\begin{circuitikz}[american, scale=1.0]
    \ctikzset{logic ports=ieee}
    \ctikzset{flipflops/scale=0.8}
    
    % Common Signals
    \node (clk_in) at (-2, -1.5) {CLK};
    \node (clr_in) at (-2, -2) {Clear};
    %\node (serial_in) at (-2, 0) {Serial In};
    
    % Positioning
    \def\ffdist{4.0}
    
    \foreach \i in {3,2,1,0} {
        \pgfmathsetmacro{\xpos}{(3-\i)*\ffdist}
        
        % Flip-Flop
        \node[flipflop D] (FF\i) at (\xpos, 0) {};
        \node[above] at (FF\i.n) {$D_\i$};
        \draw (FF\i.pin 6) -- ++(0.5, 0) coordinate (Q\i_out) node[right] {};
        
        % Clock
        \draw (FF\i.pin 3) -- (FF\i.pin 3 |- clk_in) node[circ] {};
        % Clear
        \draw (FF\i.s) -- ++(0,-0.2) coordinate (clr_pt);
        \draw (clr_pt) -- (clr_pt |- clr_in) node[circ] {};
    }
    
    % Serial Input connection to D3
    \draw (FF3.pin 1) -- ++(-0.3,0) node[left] {Serial In};
    
    % Interstage Connections (Shift Right)
    % Q3 -> D2, Q2 -> D1, Q1 -> D0
    %\draw (Q3_out) -- ++(0.5, 0) -- ++(0, 0.5) -- ($(FF2.pin 1) + (-0.5, 0.5)$) -- ++(0, -0.5) -- (FF2.pin 1);
    % Actually straight line is better if aligned
    % Q output is pin 6 (top right usually), D input is pin 1 (top left).
    % If they are on same Y, direct connect?
    % The node `node[right] {$Q_\i$}` might block.
    % Let's draw below or above.
    
    % Let's redraw connections cleanly.
    % FF3.pin 6 is Q. FF2.pin 1 is D.
    % They are spaced by \ffdist.
    \draw (FF3.pin 6) -- (FF2.pin 1);
    \draw (FF2.pin 6) -- (FF1.pin 1);
    \draw (FF1.pin 6) -- (FF0.pin 1);
    
    % Serial Output from Q0
    \draw (FF0.pin 6) -- ++(0.5, 0) node[right] {Serial Out};
    
    % Finish Rails
    \draw (clk_in) -- (clk_in -| FF0.pin 3) -- ++(2, 0);
    \draw (clr_in) -- (clr_in -| FF0.s) -- ++(2, 0);;
    
\end{circuitikz}
\end{document}
