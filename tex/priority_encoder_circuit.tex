\documentclass[border=10pt]{standalone}
\usepackage[american]{circuitikz}

\begin{document}
\begin{circuitikz}
    \ctikzset{logic ports=ieee}

    % Inputs D3..D0
    % Vertical rail layout
    \foreach \i/\label in {0/D_3, 1/D_2, 2/D_1, 3/D_0} {
        \node (in_\i) at (\i*1, 12) {$\label$};
        \draw (\i*1, 11.5) -- (\i*1, 0);
    }
    % Indices: D3=0, D2=1, D1=2, D0=3
    % X coords: 0, 1.5, 3.0, 4.5

    % --- Output x = D2 + D3 ---
    \node[or port] (or_x) at (8.5, 10) {};
    \node[right] at (or_x.out) {$x$};
    
    % Inputs: D3 (rail 0), D2 (rail 1)
    \draw (0, 10 |- or_x.in 1) node[circ]{} -- (or_x.in 1);
    \draw (1, 10 |- or_x.in 2) node[circ]{} -- (or_x.in 2);


    % --- Output y = D3 + D2'D1 ---
    % We need an AND gate first for (D2' D1)
    
    % Inverter for D2
    % Place inverter near D2 rail but moving right
    %\node[not port, scale=0.7] (inv_d2) at (3.5, 6.8) {}; 
    %\draw (1.5, 7) node[circ]{} -- (inv_d2.in); % Input from D2
    
    
    % AND gate
    \node[and port] (and_y) at (6, 6.5) {};
    \draw (and_y.in 1) -- ++(-0.3,0) node[not port, anchor=east, scale=0.7] (inv_d2) {};
    
    \draw (inv_d2.in 1 -| in_1) node[circ]{} -- (inv_d2.in 1);

    % Connect Inverter out to AND in 1
    %\draw (inv_d2.out) |- (and_y.in 1);
    
    % Connect D1 (rail 2) to AND in 2
    \draw (3.0, 0 |- and_y.in 2) node[circ]{} -- (and_y.in 2);
    
    % Final OR gate for y
    \node[or port] (or_y) at (8.5, 7) {};
    \node[right] at (or_y.out) {$y$};
    
    % Input 1: D3 (rail 0)
    \draw (0, 0 |- or_y.in 1) node[circ]{} -- (or_y.in 1);
    
    % Input 2: Output of AND
    \draw (and_y.out) |- (or_y.in 2);


    % --- Output V = D0 + D1 + D2 + D3 ---
    \node[or port, number inputs=4] (or_v) at (8.5, 3) {};
    \node[right] at (or_v.out) {$V$};
    
    % Inputs: All rails
    \foreach \rail/\pin in {0/1, 1/2, 2/3, 3/4} {
        \draw (\rail*1, 0 |- or_v.in \pin) node[circ]{} -- (or_v.in \pin);
    }

\end{circuitikz}
\end{document}
