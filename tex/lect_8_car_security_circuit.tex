\documentclass[border=10pt]{standalone}
\usepackage{circuitikz}

\begin{document}
\begin{circuitikz}[american]
    \ctikzset{logic ports=ieee}
    \tikzset{d-ff/.style={flipflop, flipflop def={t1=D, t6=Q, t4=\ctikztextnot{Q}, c3=1,td=Clear
    }}}

    % Components
    %\node[or port, number inputs=3] (OR) at (0,0) {};
    %\node[and port] (AND) at (3,0) {};
    \node[d-ff] (DFF) at (6,0) {$Q$};
    \draw (DFF.pin 1) -- ++(-0.5,0) node[and port,anchor=out] (AND) {};
    \draw (AND.in 1) -- ++(-0.5,0) node[or port,anchor=out,number inputs=3] (OR) {};
    
    % Inputs
    \node[left] at (OR.in 2) (D) {$D$};
    \node[left] at (OR.in 3) (V) {$V$};
    % Input 1 is feedback
    
    % Connections
    \draw (D) -- (OR.in 2);
    \draw (V) -- (OR.in 3);
    
    % OR to AND
    % OR output to AND input 2
    
    
    % M input to AND input 1
    \node[left] at ($(OR.in 3)+(0,-0.5)$) (M) {$M$};
    \draw (M) -| (AND.in 2);
    
    
    % AND to DFF
    \draw (AND.out) -- (DFF.pin 1);
    
    % CLK
    \draw (DFF.pin 3) -- ++(0,-0.5) coordinate(tmp) -- (tmp -| M) node[left] {CLK};
    
    % Output and Feedback
    \draw (DFF.pin 6) -- ++(1,0) node[right] {$A (Alarm)$};
    
    % Feedback Loop
    % From Q output back to OR input 1
    \draw (DFF.pin 6) -- ++(0.5,0) node[circ] (tmp) {} -- ++(0, 1.5) -| (OR.in 1);
    
    % Label Q on feedback line near OR input for clarity
    % \node[above right] at (OR.in 1) {$Q$};
    \draw (DFF.down) -- ++(0, -0.2) coordinate (tmp) -- (tmp -| M) node[left] {Reset};

\end{circuitikz}
\end{document}
