\documentclass[border=10pt]{standalone}
\usepackage{circuitikz}
\usetikzlibrary{calc}

\begin{document}
\begin{circuitikz}[american, scale=1.0]
    \ctikzset{logic ports=ieee}
    
    % Inputs
    \node (clk_in) at (-2, -5) {CLK};

   
    
    % Flip-Flops (Standard D, manually drawing Reset)
    \node[flipflop D] (DFF_A) at (8, 2) {};
    \node[above] at (DFF_A.n) {Flip-Flop A};
    \node[flipflop D] (DFF_B) at (8, -3) {};
    \node[above] at (DFF_B.n) {Flip-Flop B};

    % Logic for DA = Bx'
    % AND gate (B, x')
    \path (DFF_A.pin 1) -- ++(-3, 0) coordinate (da_pos);
    \node[and port] (AND_DA) at (da_pos) {$Bx'$};
    \draw (AND_DA.out) -- (DFF_A.pin 1);

     % x input
    
    %\draw (x_in_node) -- ++(2,0) coordinate (x_bus_top);
    %\node[circ] at (x_bus_top) {};

    % Inverter for x (to get x')
    % Place slightly lower
    \draw (AND_DA.in 2) -- ++(-1, 0) node[not port, scale=0.8, anchor=out] (NOT_x) {};
    % Route x to NOT
    \draw (NOT_x.in) -- ++(-1,0) coordinate (x_bus_top);
    \draw (x_bus_top) to[short, *-] ++(-1,0) node[left] {$x$};
    
    % Logic for DB = x
    % Direct connection from x
    \draw (DFF_B.pin 1) to[short, -*] (DFF_B.pin 1 -| x_bus_top);
    %\draw (x_bus_top) |- (DB_in) -- (DFF_B.pin 1);

    % Connect x' to AND_DA (input 2 - bottom)
    \draw (NOT_x.out) -- ++(0.5,0) |- (AND_DA.in 2);

    % Feedback B to AND_DA (input 1 - top)
    % B is at DFF_B.pin 6
    \draw (DFF_B.pin 6) -- ++(1, 0) coordinate (B_out);
    \path (DFF_B.pin 6) -- ++(0.4, 0) coordinate (B_common);
    \node[right] at (B_out) {$B$};
    \draw (B_common) to[short, *-] ++(0, 6.3) coordinate (B_top) -| (AND_DA.in 1);

    % Output Logic y = Ax (Mealy)
    % AND gate connected to A and x
    % Place to right of FF A
    \path (DFF_A.pin 6) -- ++(4, -1) coordinate (y_pos);
    \node[and port] (AND_y) at (y_pos) {$Ax$};
    
    % A to AND_y
    \draw (DFF_A.pin 6) -- ++(1, 0) coordinate (A_out);
    \node[above right] at (A_out) {$A$};
    \draw (A_out) -| (AND_y.in 1);

    % x to AND_y
    % Route x all the way
    \draw (x_bus_top) -- ++(0, -7) coordinate (x_bus_bot) -- ++(12, 0) coordinate (x_far_right) |- (AND_y.in 2);
    
    % Output y
    \draw (AND_y.out) -- ++(1, 0) node[right] {$y$};

    % Clock Distribution
    \draw (DFF_B.pin 3) -- ++(-0.5, 0) node[circ] (clk_node) {};
    \draw (clk_in) -| (clk_node);
    \draw (clk_node) |- (DFF_A.pin 3);
    
    % Reset Signal (Active Low Bubble at bottom)
    \node (rst_in) at (-2, -5.5) {Reset};
    
    % Draw bubbles at FF bottom (manually)
    \draw [fill=white] (DFF_A.s) circle(2pt);
    \draw [fill=white] (DFF_B.s) circle(2pt);
    
    % Route Reset
    % Reset line (rst_in) -> horizontal -> vertical bus -> tap off to FFs.
    % Vertical bus at x=6.5
    
    % Main route (Reset to B)
    \draw (rst_in) -| ($(DFF_B.s) + (0, -2pt)$);
    
    % Route to A
    \draw ($(DFF_A.s) + (0, -2pt)$) -- ++(0, -0.5) -- ++(-1.2,0) coordinate (rst_out);
    \draw (rst_out) to[short, -*] (rst_out |- rst_in);
    % Top of bus is just a corner, no dot needed unless we continued up.


\end{circuitikz}
\end{document}
