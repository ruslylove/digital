\documentclass[tikz,border=10pt]{standalone}
\usepackage{tikz}
\usetikzlibrary{arrows.meta, positioning, calc}

\begin{document}

% Define fonts and colors
\renewcommand{\familydefault}{\sfdefault}
\definecolor{analogColor}{HTML}{D62728}
\definecolor{digitalColor}{HTML}{1F77B4}
\definecolor{gridColor}{HTML}{E0E0E0}

\begin{tikzpicture}[
    axis arrow/.style={->, >=Latex, thick, black!80},
    analog line/.style={analogColor, very thick, smooth, line cap=round},
    digital line/.style={digitalColor, very thick},
    sample point/.style={fill=digitalColor, circle, inner sep=1.5pt},
    title label/.style={font=\large\bfseries, align=center},
    anno label/.style={font=\small, black!70, align=center},
    grid line/.style={draw=gridColor, thin, dashed}
]

% --- Define the signal function ---
\pgfmathdeclarefunction{signalFunc}{1}{%
  \pgfmathparse{2.5 + 1.2*sin(deg(#1*1.5)) + 0.7*sin(deg(#1*4))}%
}

% Dimensions
\def\xAxisLen{6.5}
\def\yAxisLen{5}
\def\plotWidth{6}
\def\plotHeight{4.5}

%=========================
% Plot 1: Analog Signal
%=========================
\begin{scope}[local bounding box=analogScope]
    \node[title label] at (\plotWidth/2, \yAxisLen + 1.8) {Analog Signal};
    \node[anno label] at (\plotWidth/2, \yAxisLen + 1.3) {(Continuous Time \& Amplitude)};


    % Grid
    \draw[grid line] (0,0) grid[xstep=1cm, ystep=1cm] (\plotWidth,\plotHeight);

    % Axes
    \draw[axis arrow] (0,0) -- (\xAxisLen,0) node[right] {Time ($t$)};
    \draw[axis arrow] (0,0) -- (0,\yAxisLen) node[above] {Amplitude ($A(t)$)};

    % Curve
    \draw[analog line] plot[domain=0:\plotWidth, samples=200] (\x, {signalFunc(\x)});

    \node[anno label, below=0.8cm of analogScope.south] {
        Signal changes smoothly over time.\\
        Can take any value within a range.
    };
\end{scope}

%=========================
% Plot 2: Digital Signal
%=========================
\begin{scope}[xshift = \xAxisLen cm + 3cm, local bounding box=digitalScope]
    \node[title label] at (\plotWidth/2, \yAxisLen + 1.8) {Digital Signal};
    \node[anno label] at (\plotWidth/2, \yAxisLen + 1.3) {(Discrete Time \& Amplitude)};

    % Quantization Grid
    \foreach \y in {0.5, 1.0, ..., \plotHeight} { \draw[gridColor, thin] (0,\y) -- (\plotWidth,\y); }
    \foreach \x in {0.5, 1.0, ..., \plotWidth} { \draw[gridColor, very thin, dotted] (\x,0) -- (\x,\plotHeight); }

    % Axes
    \draw[axis arrow] (0,0) -- (\xAxisLen,0) node[right] {Time ($nT$)};
    \draw[axis arrow] (0,0) -- (0,\yAxisLen) node[above] {Amplitude ($D[n]$)};

    % Faint background analog reference
    \draw[analog line, opacity=0.2, thin] plot[domain=0:\plotWidth, samples=200] (\x, {signalFunc(\x)});

    % --- 1. Draw Sample Points (Dots) ---
    % We do this loop separately to place the dots
    \def\samplingPeriod{0.5}
    \foreach \t in {0, \samplingPeriod, ..., \plotWidth} {
        \node[sample point] at (\t, {round(signalFunc(\t)*2)/2}) {};
    }

    % --- 2. Draw the Step-wise Signal (The Line) ---
    % FIX: Use 'const plot' style instead of a loop inside the path.
    % 'const plot' draws horizontal then vertical lines automatically.
    \draw[digital line] plot[
        const plot, 
        domain=0:\plotWidth, 
        samples=13  % (6.0 / 0.5) + 1 = 13 samples
    ] (\x, {round(signalFunc(\x)*2)/2});

    \node[anno label, below=0.8cm of digitalScope.south] {
        Signal represented by discrete steps.\\
        Sampled at specific times ($T=0.5$).
    };

    % Annotations
    \draw[<->, thin, black!60] (1, -0.3) -- (1.5, -0.3) node[midway, below, font=\scriptsize] {$T$};
    \draw[dotted, thin, black!60] (1,0) -- (1, {signalFunc(1)});
    \draw[dotted, thin, black!60] (1.5,0) -- (1.5, {signalFunc(1.5)});

\end{scope}

\end{tikzpicture}
\end{document}