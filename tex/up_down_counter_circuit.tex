\documentclass[border=10pt]{standalone}
\usepackage{circuitikz}

\begin{document}
\begin{circuitikz}[american, scale=1.2]
    \ctikzset{logic ports=ieee}
    \tikzset{
        jk-ff/.style={flipflop, flipflop def={
            t1=J, t3=K, t6=Q, t4=\ctikztextnot{Q},
            c2=1 % Clock at pin 2 (middle left)
        }}
    }

    % Input x
    \node (x) at (-2, 2) {$x$};

    % Flip Flop 0 (Q0) - LSB
    \node[jk-ff] (FF0) at (2, 0) {$Q_0$};
    
    % Flip Flop 1 (Q1) - MSB
    \node[jk-ff] (FF1) at (8, 0) {$Q_1$};

    % Clock
    \node (clk) at (0, -3) {CLK};
    \draw (clk) -| (FF0.pin 2);
    \draw (clk) -| (FF1.pin 2);

    % Logic for FF0: J=1, K=1
    \node[vcc] (vcc) at (0, 1) {1};
    \draw (vcc) -| (FF0.pin 1); % J0
    \draw (vcc) -| (FF0.pin 3); % K0

    % Logic for FF1: J=K= x XNOR Q0
    % Need Q0 output from FF0
    \draw (FF0.pin 6) -- ++(1,0) coordinate (Q0_out);
    \node[right] at (Q0_out) {$Q_0$}; % Label output
    
    % XNOR Gate
    % Inputs: x and Q0
    \node[xnor port] (xnor) at (5, 1.5) {};
    
    % Wire x to XNOR
    \draw (x) |- (xnor.in 1);
    
    % Wire Q0 to XNOR
    \draw (Q0_out) |- (xnor.in 2);
    
    % Output of XNOR to J1 and K1
    \draw (xnor.out) -- ++(0.5,0) coordinate (logic_out);
    \draw (logic_out) |- (FF1.pin 1); % J1
    \draw (logic_out) |- (FF1.pin 3); % K1

    % FF1 Output
    \draw (FF1.pin 6) -- ++(1,0) node[right] {$Q_1$};

\end{circuitikz}
\end{document}
