\documentclass[border=10pt]{standalone}
\usepackage{circuitikz}
\usetikzlibrary{calc}

\begin{document}
\begin{circuitikz}[american]
    \ctikzset{logic ports=ieee}

    % NAND Latch part (Same position as gated latch)
    \draw (6,2) node[nand port] (nand1) {};
    \draw (6,0) node[nand port] (nand2) {};

    % Gating NAND gates
    % Aligned roughly with latch inputs
    \draw (nand1.in 1) -- ++(-1,0) node[nand port, anchor=out] (ng1) {};
    \draw (nand2.in 2) -- ++(-1,0) node[nand port, anchor=out] (ng2) {};

    % Connect Gating to Latch
    \draw (ng1.out) -- (nand1.in 1);
    \draw (ng2.out) -- (nand2.in 2);

    % Control Input (C)
    \draw (ng1.in 2) -- ++(-0.5,0) coordinate(c_top);
    \draw (ng2.in 1) -- ++(-0.5,0) coordinate(c_bot);
    \draw (c_top) -- (c_bot);
    \draw ($(c_top)!0.5!(c_bot)$) -- ++(-2.5,0) node[left] {$C$};
    \node[circ] at ($(c_top)!0.5!(c_bot)$) {}; 

    % Input D Logic
    % D goes to ng1.in 1 (S input equivalent)
    % D also goes through Inverter to ng2.in 2 (R input equivalent)
    
    % Draw D input node further left
    \draw (ng1.in 1) -- ++(-3,0) coordinate(d_in_line) node[left] {$D$};
    \node[] at (d_in_line) {}; % Connection point for split

    % Inverter for D -> R
    % Place inverter on the path to ng2.in 2.
    % Scaled down as requested (scale=0.7)
    
    \draw (ng2.in 2) -- ++(-1,0) node[not port, scale=0.5, anchor=out] (inv) {};
    
    % Connect D to Inverter input
    % D is at y=2 approx (ng1.in 1).
    % Tap off D line.
    \draw (d_in_line) -- ++(0.5,0) coordinate(splitD);
    \draw (splitD) |- (inv.in);
    \node[circ] at (splitD) {};
    
    % Feedback Cross-coupling (Standard NAND Latch)
    % Top Out -> Bot In 1
    \draw (nand1.out) -- ++(0.5,0) coordinate(Qout);
    \draw (Qout) -- ++(0, -0.5) -- ($(nand2.in 1) + (0, 0.5)$) -- (nand2.in 1);
    
    % Bot Out -> Top In 2
    \draw (nand2.out) -- ++(0.5,0) coordinate(Qpout);
    \draw (Qpout) -- ++(0, 0.5) -- ($(nand1.in 2) + (0, -0.5)$) -- (nand1.in 2);
    
    % Outputs
    \draw (Qout) -- ++(1,0) node[right] {$Q$};
    \draw (Qpout) -- ++(1,0) node[right] {$Q'$};
    
    \node[circ] at (Qout) {};
    \node[circ] at (Qpout) {};

\end{circuitikz}
\end{document}
