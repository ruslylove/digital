\documentclass[border=10pt]{standalone}
\usepackage{circuitikz}
\usetikzlibrary{calc}

\begin{document}
\begin{circuitikz}[american, scale=1.0]
    \ctikzset{logic ports=ieee}
    
    % Inputs - Leftmost
    \node (clk_in) at (0, -5) {CLK};
    
    % Flip-Flops - Far Right
    \node[flipflop JK] (JK_A) at (10, 2) {};

    \node[flipflop JK] (JK_B) at (10, -3) {};



    
    
    
    % Logic for JA = B
    % Direct connection from B (pin 1 of JK_B is J, wait, outputs are Q/Qbar)
    % B is JK_B.pin 6
    % Route B to JA (JK_A.pin 1)
    \draw (JK_A.pin 1) -- ++(-1, 0) coordinate (JA_in);
    % Connection will handle in Feedback section
    
    % Logic for KA = Bx'
    % AND gate (B, x')
    \path (JK_A.pin 3) -- ++(-3, 0) coordinate (ka_pos);
    \node[and port, anchor=out] (AND_KA) at (ka_pos) {$Bx'$};
    \draw (AND_KA.out) -- (JK_A.pin 3);

    % Input x
    % Need a bus for x
    \draw (AND_KA.in 2) to[short, -*] ++(-1, 0) coordinate (x_bus_top);
    %\node (x_in_node) at (-1, 0) {$x$};
    %\draw (x_in_node) -- ++(2,0) coordinate (x_bus_top);
    \node[circ] at (x_bus_top) {};

    % Inverter for x (x')
    % Positioned early to be available
    \draw (x_bus_top) -- ++(-0.2, 0) node[not port, anchor=out, scale=0.5] (INV_x) {};
    \draw (INV_x.in) -- ++(-1,0) node[left] {$x$};
    \path (INV_x.in) -- ++(-0.2, 0) coordinate (x_bar_bus);
    
    % Logic for JB = x'
    % Direct connection from x'
    \draw (JK_B.pin 1) -| (x_bus_top);
    
    % Logic for KB = A XOR x
    % XOR gate (A, x)
    \path (JK_B.pin 3) -- ++(-3, 0) coordinate (kb_pos);
    \node[xor port, anchor=out] (XOR_KB) at (kb_pos) {$A \oplus x$};
    \draw (XOR_KB.out) -- (JK_B.pin 3);
    
    % Wiring Inputs
    % JB = x'
    % Connect x_bar_bus to JB_in
    % x_bar_bus is around (0, -1.5). JB_in is around (9, -3 + offset).
    % Let's extend x_bar_bus downwards
    %\draw (x_bar_bus) -- ++(0, -5) coordinate (x_bar_low);
    
    % Connect x' to JB
   % \draw (x_bar_bus |- JB_in) to[short, *-] (JB_in) -- (JK_B.pin 1);
    
    % Connect x' to AND_KA (input 2)
    %\draw (x_bar_bus |- AND_KA.in 2) to[short, *-] (AND_KA.in 2);
    
    % Connect x to XOR_KB (input 2)
    % x_bus is at x=0
    \draw (x_bar_bus) to[short, *-] ++(0, -1) |- (XOR_KB.in 2);
    
    % Feedback Connections
    % A (JK_A.pin 6)
    \draw (JK_A.pin 6) -- ++(1, 0) coordinate (A_out);
    \node[right] at (A_out) {$A$};
    \path (JK_A.pin 6) -- ++(0.2, 0) coordinate (A_com);
    % A to XOR_KB (input 1)
    % Loop under or over? KB is bottom.
    % Route A_out down and left.
    \draw (A_com) to[short, *-] ++(0, -3) -| (XOR_KB.in 1);
    
    % B (JK_B.pin 6)
    \draw (JK_B.pin 6) -- ++(1, 0) coordinate (B_out);
    \node[right] at (B_out) {$B$};
    \path (JK_B.pin 6) -- ++(0.6, 0) coordinate (B_com);
    
    % B to JA (JK_A.pin 1)
    % Loop UP and left
    \draw (B_com) to[short, *-] ++(0, 6.3) coordinate (B_top) -| (AND_KA.in 1);
    
    % B to AND_KA (input 1)
    % Tap off B_top line?
    % B_top is at x=11, y=3. JK_A.pin 1 is y=2.2 (approx).
    % AND_KA is at y=1.7 (approx).
    \draw (JK_A.pin 1) to[short, -*] (JK_A.pin 1 -| AND_KA.in 1);
    %\draw (B_top -| AND_KA.in 1) to[short, *-] (AND_KA.in 1);
    
    % Clock Distribution
    \draw (JK_B.pin 2) -- ++(-0.5, 0) node[circ] (clk_node) {};
    \draw (clk_in) -| (clk_node);
    \draw (clk_node) |- (JK_A.pin 2);

\end{circuitikz}
\end{document}
