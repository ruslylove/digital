\documentclass[border=10pt]{standalone}
\usepackage{circuitikz}
\usetikzlibrary{calc}

\begin{document}
\begin{circuitikz}[american, scale=1.0]
    \ctikzset{logic ports=ieee}
    \ctikzset{flipflops/scale=0.85}
    
    % Signals
    \node (clk_in) at (-3, 0) {Clock};
    \node (logic_1) at (-3, -3) {Logic 1};
    
    % Flip-Flops
    % LSB (A0) on Left -> MSB (A3) on Right
    % FF0: J=1, K=1
    % FF1: J=Q3', K=1
    % FF2: J=1, K=1
    % FF3: J=Q1*Q2, K=1
    
    \def\ffdist{4.5}
    
    \foreach \i [count=\n from 0] in {0,1,2,3} {
        \pgfmathsetmacro{\xpos}{\n*\ffdist}
        
        % JK Flip-Flop
        \node[flipflop JK,flipflop def={nd=1,n2=1}] (FF\i) at (\xpos, 0) {};
        
        % Outputs
        \node[above] at (FF\i.n) {$FF_\i$};
        \draw (FF\i.pin 6) -- ++(0.5, 0) coordinate (Q\i_out);
       % \draw (Q\i_out) -- ++(0.3, 0) node[circ, label={above:$Q_\i$}] {};
        
        % Tie K to Logic 1 for all
        \draw (FF\i.pin 3) -- ++(-0.2, 0) coordinate (K_stub_\i);
        \ifnum\i=3
        \else
        \draw (K_stub_\i) -- (K_stub_\i |- logic_1) node[circ] {};
        \fi
    }
    
    % Logic 1 Rail (Top)
    \draw (logic_1) -| (K_stub_3);
    
    % Clock Connections 
    % A0 Clock -> External Clock
    \draw (clk_in) -- (FF0.pin 2);
    
    % FF1 Clock <- Q0
   % \draw (Q0_out) -- ++(0, -3.2) coordinate (rip0);
    \draw (Q0_out) |- (FF1.pin 2);
    
    % FF2 Clock <- Q1
    \draw (Q1_out) |- (FF2.pin 2);
    %\draw (rip1) -- (rip1 -| FF2.pin 2) -- (FF2.pin 2);
    
    % FF3 Clock <- Q0 (Directly from Q0)
    % According to design, FF3 is clocked by Q0? 
    % Wait, standard design:
    % Q0 -> CP1
    % Q1 -> CP2
    % Q0 -> CP3 (Falling edge of Q0 clocks Q3)
    
    %\draw (rip0) -- (rip0 -| FF3.pin 2) -- (FF3.pin 2);
    \draw (FF3.pin 2) -- ++(-1,0) -- ++(0, -2.5) -| (Q0_out |- FF1.pin 2) node[circ] {} ;
    
    
    % J Inputs Logic
    
    % FF0: J=1
    \draw (FF0.pin 1) -- ++(-0.2, 0) coordinate (J0_stub);
    \draw (J0_stub) -- (J0_stub |- FF0.pin 3) node[circ] {};
    
    % FF1: J = Q3'
    % Need Q3' output from FF3.
    % Connect Q3' to J1.
    \draw (FF3.pin 4) -- ++(0.2, 0) -- ++(0, -1) coordinate (Q3n_turn);
    \draw (Q3n_turn) -- ($(Q3n_turn -| FF1.pin 1) + (-0.5,0)$) |- (FF1.pin 1);
    
    % FF2: J=1
    \draw (FF2.pin 1) -- ++(-0.2, 0) coordinate (J2_stub);
    \draw (J2_stub) -- (J2_stub |- FF2.pin 3) node[circ] {};
    
    % FF3: J = Q2 AND Q1
    % Place AND gate before FF3 J input
    \node[and port,scale=0.8] (AND) at ($(FF3.pin 1) + (-1, 0)$) {};
    \draw (AND.out) -- (FF3.pin 1);
    
    % Inputs to AND: Q2 and Q1
    % Since AND is near FF3, Q2 is close. Q1 is further.
    
    % Connect Q2 to AND input 1 (top)
    \draw (Q2_out) ++(0.2,0) coordinate (tap_Q2);
    
    \draw (Q2_out) -| (AND.in 2);
    
    % Connect Q1 to AND input 2 (bottom)
    % Run a line below or above? Above might cross Logic 1 rail. Below implies crossing everything.
    % Let's run it midway or via top.
    

    % Route via top area but below logic 1 rail
    \draw (Q1_out) to[short,*-] ++(0, +1) coordinate (route_q1);
    \draw (route_q1) -- (route_q1 -| AND.in 1) -- (AND.in 1);
    
    % No explicit reset lines needed for this design (it resets via J/K logic)
    % Remove old reset logic and pins?
    % The user didn't ask to remove reset pins, but "reset logic" usually means the NAND gate stuff is gone.
    % I will remove the NAND gate reset logic completely.
    
\end{circuitikz}
\end{document}
