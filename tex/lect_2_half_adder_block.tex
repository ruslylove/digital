\documentclass[border=10pt]{standalone}
% We need tikz to draw shapes, and circuitikz loads tikz automatically.
\usepackage{circuitikz}
% We load the 'calc' library to help easily calculate positions along the edges of the box.
\usetikzlibrary{calc}

\begin{document}

\begin{circuitikz}
    % ============================
    % 1. Define Styles
    % ============================
    % Create a style for the main block box so it's easy to change later
    \tikzset{block/.style={
        draw,               % Draw the border
        rectangle,          % Shape type
        minimum height=3cm, % Taller than wide
        minimum width=2.5cm,% Width
        align=center,       % Center text inside
        fill=white,         % Background color
        thick               % Thicker border line
    }}

    % ============================
    % 2. Draw the Main Block
    % ============================
    % Place the node at coordinate (0,0) and name it (HA)
    \node[block] (HA) at (0,0) {\large Half\\Adder};


    % ============================
    % 3. Define Connection Points
    % ============================
    % We calculate points along the left edge (west) and right edge (east).
    % "pos=0.75" means 75% of the way up the edge.
    
    % Inputs on the left side
    \coordinate (inX) at ($(HA.south west)!0.75!(HA.north west)$);
    \coordinate (inY) at ($(HA.south west)!0.25!(HA.north west)$);

    % Outputs on the right side
    \coordinate (outS) at ($(HA.south east)!0.75!(HA.north east)$);
    \coordinate (outC) at ($(HA.south east)!0.25!(HA.north east)$);


    % ============================
    % 4. Draw Wires and Labels
    % ============================
    % Input Wires (using [<-] to draw an arrow pointing left into the box)
    % We draw from the anchor point on the box, extend 1.5cm to the left.
    \draw[-, thick] (inX) -- ++(-1.5, 0) node[left] {\large $x$};
    \draw[-, thick] (inY) -- ++(-1.5, 0) node[left] {\large $y$};

    % Output Wires (using [->] to draw an arrow pointing right out of the box)
    \draw[-, thick] (outS) -- ++(1.5, 0) node[right] {\large $s$ (sum)};
    \draw[-, thick] (outC) -- ++(1.5, 0) node[right] {\large $c$ (carry)};

\end{circuitikz}

\end{document}