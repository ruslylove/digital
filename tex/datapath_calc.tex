\documentclass[border=10pt]{standalone}
\usepackage{tikz}
\usepackage{circuitikz}
\usetikzlibrary{positioning, calc}

\begin{document}
\begin{circuitikz}[
    thick,
    font=\sffamily,
    arrow/.style={-{Latex}, line width=1pt},
    label_text/.style={font=\footnotesize}
]

    % Registers B and C (Top)
    % Use flipflop but rotate -90 so D is Up, Q is Down.
    % Label externally to avoid rotation issues.
    \node[flipflop D, rotate=-90, fill=blue!10] (reg_b) {};
    \node[above=0.2cm of reg_b] {Register B};
    
    \node[flipflop D, rotate=-90, fill=blue!10, right=2.5cm of reg_b] (reg_c) {};
    \node[above=0.2cm of reg_c] {Register C};

    % Adder (Middle)
    % Rotate -90 so Inputs West->Up, Output East->Down.
    \node[adder, fill=yellow!10, rotate=-90, below=2cm of $(reg_b)!0.5!(reg_c)$] (alu) {};
    \node[right=0.1cm of alu, font=\footnotesize] {Add};

    % Register A (Bottom)
    \node[flipflop D, rotate=-90, fill=blue!10, below=2cm of alu] (reg_a) {};
    \node[below=0.2cm of reg_a] {Register A};

    % Connections
    % Reg B Q (pin 6) -> Adder Input 1
    % Q (East) rotated -> South (Screen Down).
    % Adder Input 1?
    % Connecting to specific anchors if available, or visual corners.
    % Adder West side (rotated -> North) receives inputs.
    % Let's create an explicit path.
    \draw[arrow] (reg_b.pin 6) -- ++(0,-0.8) -| ($(alu.west) + (-0.5, 0)$) -- (alu.west);
    % Actually cleaner to connect to alu corners rotated.
    % Or simply coordinate logic.
    % Reg B (Left) to Adder. Reg C (Right) to Adder.
    \draw[arrow] (reg_b.pin 6) -- ++(0,-0.5) -| ($(alu.north) + (-0.3, 0.3)$) -- (alu.north); 
    % Note: alu.north is rotated East? No.
    % Shape anchors rotate. alu.north is shape north.
    % Rotated -90: alu.north points East.
    % alu.west points North.
    % So verify: I want to hit alu.west (Screen Up).
    
    % Let's stick to using coordinates relative to center for robustness if anchors are confusing.
    % alu.center.
    \draw[arrow] (reg_b.pin 6) |- ($(alu.center) + (-0.5, 1)$) -- ($(alu.center) + (-0.5, 0)$); 
    % Wait, adder is circle. Inputs usually implicit.
    % I will draw arrows stopping at circumference.
    \draw[arrow] (reg_b.pin 6) -- ++(0,-0.8) -| ($(alu.center) + (-0.4, 0.4)$); 
    \draw[arrow] (reg_c.pin 6) -- ++(0,-0.8) -| ($(alu.center) + (0.4, 0.4)$);
    
    % Adder Output (East -> Screen Down) to Reg A D (pin 1, West -> Screen Up).
    \draw[arrow] (alu.east) -- (reg_a.pin 1);

    % Control Signal (ALoad)
    % From Screen Left to Reg A.
    % Reg A South (Screen West).
    % Let's assume ALoad enables Clock or is a specific enable.
    % Flipflop has pin 3 (clk)?
    % We will just draw arrow to the side.
    \draw[arrow, color=red] ($(reg_a) + (-3, 0)$) -- node[above, font=\scriptsize] {ALoad} (reg_a.south);

\end{circuitikz}
\end{document}
