\documentclass[border=10pt]{standalone}
\usepackage{tikz}
\usepackage{circuitikz}
\ctikzset{logic ports=ieee}
\usetikzlibrary{positioning, calc, decorations.markings, shapes.geometric}

% Define the register style similar to counter_datapath.tex
\tikzset{register_style/.style={muxdemux, muxdemux def={Lh=3, Rh=3, NL=2, NB=1, NT=1, w=5.0, NR=0},
                            muxdemux label={L1=Load, L2=Clear, L3=Clock, T1=$D$, B1=$Q$}}}

\begin{document}

\begin{circuitikz}[
    font=\sffamily,
    arrow/.style={-Latex, line width=1pt},
    label_text/.style={font=\footnotesize},
    bus/.style={-Latex, line width=1.5pt,
        postaction={decorate},
        decoration={markings, mark=at position 0.5 with {
            \draw[thick,-] (-2pt,-3pt) -- (2pt,3pt);
            \node[above left=0.5pt] {\footnotesize $4$}; 
        }}},
    control_signal/.style={-Latex, line width=0.5pt},
]

    % --- Register A ---
    \node[register_style, muxdemux label={L1=ALoad, L2=, L3=Clk, T1=In}] (RegA) at (0,4) {Reg A};
    
    % Input arrow to Reg A
    \draw[bus] ($(RegA.btpin 1)+(0,1)$) -- (RegA.btpin 1);

    % --- Comparator (A vs 5) ---
    % Using a muxdemux block for comparator
    \node[muxdemux, muxdemux def={Lh=2, Rh=2, NL=0, NB=0, NT=2, w=2.5, NR=1},
          muxdemux label={T1=$A$, T2=$B$, R1={$= $}}, anchor=btpin 1] (Comp) at (0, 0) {Comp};
    
    % Connect Reg A to Comp A input (tpin 1 -> btpin 1)
    \draw[bus] (RegA.bbpin 1) -- ++(0,-1.5) -| (Comp.btpin 1);
    
    % Connect Constant 5 to Comp B input (tpin 2 -> btpin 2)
    \draw[bus] ($(Comp.btpin 2)+(0,1)$) node[above] {$'5'$} -- (Comp.btpin 2);
    
    % Status Signal
    \draw[control_signal] (Comp.brpin 1) -- ++(1.5,0) node[right, label_text] {$(A=5)$};

    % --- Mux (Trapezoidal 2-to-1 using muxdemux) ---
    % Lh=6 (Left->Top Height), Rh=3 (Right->Bottom Height) -> Trapezoid
    % rotate=-90: Left->Top, Right->Bottom, Bottom->Left.
    % NL=2 (In), NR=1 (Out), NB=1 (Sel -> Left).
    \node[muxdemux, muxdemux def={Lh=6, Rh=3, NL=2, NB=1, NT=0, w=2, NR=1},
          muxdemux label={L1=$0$, L2=$1$, B1=$Sel$},
          rotate=-90] (Mux) at (6, 4) {Mux};
          
    % Mux Inputs (Top -> Left pins)
    \draw[bus] ($(Mux.blpin 1)+(0,1.5)$) node[left] {$'8'$} -- (Mux.blpin 1);
    \draw[bus] ($(Mux.blpin 2)+(0,1.5)$) node[left] {$'13'$} -- (Mux.blpin 2);

    % --- Register B ---
    \node[register_style, muxdemux label={L1=BLoad, L2=, L3=Clk}] (RegB) at (6, 0) {Reg B};
    
    % Connect Mux Out (Bottom -> Right pin) to Reg B In
    \draw[bus] (Mux.brpin 1) -- (RegB.btpin 1);

    % --- Control Signals routed from Left ---
    \def\controlX{-3}
    
    % ALoad
    \draw[control_signal] (\controlX, 0 |- RegA.blpin 1) node[left, label_text] {$ALoad$} -- (RegA.blpin 1);
    
    % BLoad
    \draw[control_signal] (\controlX, -2) node[left, label_text] {$BLoad$} -- ++(7,0) |- (RegB.blpin 1);
    
    % Muxsel
    % Route to West side of Mux (Bottom pin 1 of muxdemux, rotated -90 -> Left)
    \draw[control_signal] (\controlX, 2.5) node[left, label_text] {$Muxsel$} -| (Mux.bpin 1) -- (Mux.bbpin 1);
    
    % --- Tri-state Buffer ---
    % Placed below Register B
    % rotate=-90: In(Top), Out(Bottom), Control(Right).
    % yscale=-1: Flips Y axis (before rotation?). 
    % Let's use yscale=-1 with rotate=-90.
    % If we flip Y, Up becomes Down (relative). Rotated -90 -> Left.
    \node[buffer port, rotate=-90, yscale=-1, scale=0.5] (Tri) at (6, -3) {};
    
    % Connect Reg B Out to Tri-state In
    \draw[bus] (RegB.bbpin 1) -- (Tri.bin);
    
    % Tri-state Output
    \draw[bus] (Tri.bout) -- ++(0,-1) node[below] {$Output$};
    
    % Tri-state Control Signal (Out)
    % With yscale=-1, 'up' should be on the Left.
    \draw[control_signal] (\controlX, 0 |- Tri.up) node[left, label_text] {$Out$} -- (Tri.up);
    
    % Clock Line (Common)
    \draw[control_signal] (\controlX, 0 |- RegA.blpin 2) node[left, label_text] {Clock} -- (RegA.blpin 2);
    
    \path (RegA.lpin 2) ++ (-0.5,0) coordinate (clk_jct) {};
    \draw[control_signal] (clk_jct) to[short,*-] ++(0,-1.5) -| (RegB.lpin 2) -- (RegB.blpin 2);
   % \fill [black] (clk_jct) circle (2pt);

\end{circuitikz}
\end{document}
