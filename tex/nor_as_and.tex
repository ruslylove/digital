\documentclass[border=10pt]{standalone}
\usepackage{circuitikz}

\ctikzset{
logic ports=ieee,
logic ports/scale=0.7,
}

\begin{document}
\begin{circuitikz}[]

    % === Input Inverters ===
    % Inverter for Input A (Top left)
    \draw (0, 1) node[nor port] (NOTA) {};
    % Wire A to both inputs
    \draw (NOTA.in 1) -- ++(-0.5,0) coordinate(splitA) -- ++(-0.5,0) node[left] {$A$};
    \draw (splitA) node[circ]{} |- (NOTA.in 2);

    % Inverter for Input B (Bottom left)
    \draw (0, -1) node[nor port] (NOTB) {};
    % Wire B to both inputs
    \draw (NOTB.in 1) -- ++(-0.5,0) coordinate(splitB) -- ++(-0.5,0) node[left] {$B$};
    \draw (splitB) node[circ]{} |- (NOTB.in 2);


    % === Final NOR Gate ===
    \draw (3, 0) node[nor port] (FINAL) {};


    % === Connections ===
    % Connect Inverter A out to Final Top In
    \draw (NOTA.out) -| (FINAL.in 1);
    % Connect Inverter B out to Final Bottom In
    \draw (NOTB.out) -| (FINAL.in 2);


    % === Output ===
    \draw (FINAL.out) -- ++(1,0) node[right] {$Y = A \cdot B$};

\end{circuitikz}
\end{document}