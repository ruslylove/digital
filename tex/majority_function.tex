\documentclass{article}
\usepackage[a4paper,margin=1in]{geometry}
\usepackage{amsmath}
\usepackage{amssymb}
\usepackage{graphicx}
\usepackage{circuitikz}
\usepackage{listings}

\title{3-Input Majority Function}
\author{}
\date{}

\begin{document}
\maketitle

\section*{Problem Description}
A majority function is a Boolean function that is true when more than half of its inputs are true. This document illustrates the design of a 3-input majority function.

\section*{1. Truth Table}

\begin{center}
\begin{tabular}{|c|c|c||c|}
\hline
\multicolumn{3}{|c||}{Inputs} & Output \\
\hline
\textit{x} & \textit{y} & \textit{z} & \textit{f} \\
\hline
0 & 0 & 0 & 0 \\
0 & 0 & 1 & 0 \\
0 & 1 & 0 & 0 \\
0 & 1 & 1 & 1 \\
1 & 0 & 0 & 0 \\
1 & 0 & 1 & 1 \\
1 & 1 & 0 & 1 \\
1 & 1 & 1 & 1 \\
\hline
\end{tabular}
\end{center}

\section*{2. Boolean Expression}

The Sum-of-Products (SOP) expression derived from the truth table is:
$f = x'yz + xy'z + xyz' + xyz$

This can be simplified using Boolean algebra:
$f = yz(x' + x) + xz(y' + y) + xy(z' + z)$
$f = yz + xz + xy$

The simplified expression is:
$f = xy + yz + xz$

\section*{3. Circuit Diagram}

The logic circuit for the simplified expression $f = xy + yz + xz$.

\begin{figure}[h!]
\centering
\begin{circuitikz}
    \draw
    (0,4) node[and port] (and1) {}
    (0,2) node[and port] (and2) {}
    (0,0) node[and port] (and3) {}
    (2,2) node[or port, number of inputs=3] (or1) {}
    (and1.out) -- (or1.in 1)
    (and2.out) -- (or1.in 2)
    (and3.out) -- (or1.in 3)
    (or1.out) -- (3,2) node[right] {$f$};

    \node at (-2,4.25) (x1) {x};
    \node at (-2,3.75) (y1) {y};
    \draw (x1) -- (and1.in 1);
    \draw (y1) -- (and1.in 2);

    \node at (-2,2.25) (y2) {y};
    \node at (-2,1.75) (z1) {z};
    \draw (y2) -- (and2.in 1);
    \draw (z1) -- (and2.in 2);

    \node at (-2,0.25) (x2) {x};
    \node at (-2,-0.25) (z2) {z};
    \draw (x2) -- (and3.in 1);
    \draw (z2) -- (and3.in 2);
\end{circuitikz}
\caption{Circuit for the 3-input majority function.}
\end{figure}

\end{document}
